\chapter{Input\label{ch:input}}
This chapter describes the input files of \neotwo and their formats.
Note that this is neither considered to be a reference, nor to give a
detailed description of the formats/input files.\\
The chapter is splited into two sections, the first for the parallel
version, the second for the quasilinear version. Paragraphs within each
section list the input files, that might be necessary, in alphabetical
order by filename/type.

\section{PAR}


\section{QL}


\paragraph{boozer}
Read in subroutine \codevariable{neo\_read} in \file{neo\_sub.f90} (only
the axisymetric part?).

\paragraph{efit}
Efit files are read in \file{field\_divB0} in the routines
\codevariable{read\_dimeq1} (basic size information) and
\codevariable{read\_eqfile1} (all the data, and again the basic size
information).

For efit case, input is assumed to be in SI units. The subroutine
\codevariable{field\_eq} (which calls the reading routines above), will
then convert the values to CGS units.

\paragraph{field\_divB0.inp}
A fixed format file.

More is not known at the moment.

\paragraph{neo.in}
A fixed format file.

Most important parameter is probably the name of the file which contains
the magnetic background.

Read in subroutine \codevariable{neo\_read\_control} in
\file{neo\_sub.f90}.

There is an example file with all the parameters in the doc folder,
which should also document the parameters.

\paragraph{neo2.in}
A namelist file.
Read in main file.

Memory scales at least with cube of lag parameter, as the number of
non-zero elements of the sparse matrix scales with the cube of number of
basis polynoms.

There is an example file with all the parameters in the doc folder,
which should also document the parameters.

See Fig.~\ref{fig:collop_bspline_dist} for example of the effect of the
parameter \inputparameter{collop\_bspline\_dist}.

\begin{figure}
  \centering{}
  %~ \includegraphics[width=0.48\textwidth]{../../Images/documentation_bspline_distance}
  \caption{Visualize the effect of the paramter collop\_bspline\_dist on
  the velocity grid. The different cases are drawn at different vertical
  positions. For the values ``= 1.5'' and ``=2'', there are two cases with
  a different number of points.}
  \label{fig:collop_bspline_dist}
\end{figure}
The figure shows at different vertical levels different cases for the
parameter.
For values $< 1$ the distance between the points will decrease with
increasing velocity. The case $= 1$ corresponds to an equidistant grid.
For values $> 1$ the distance between the points will increase with
increasing velocity.
Note that the image of the case $a < 1$ correlates to the case $1/a > $:
the points are mirrored at the center.
For the cases with collop\_bspline\_dist $=1.5$ and $=2$ there are two
different cases. One with five points and one with nine points. For the
case with collop\_bspline\_dist $=1$ this grid is easy to imagine, as
the distance between points will just be halved.
In the case $> 1$ the additional points lead mainly to an increase of
resolution at low velocities.
What can not be seen directly from the figure, is that for
collop\_bspline\_dist $> 1$, there is a region at high velocities, in
which you can not get any points, regardless on how large the number of
knots would be choosen. For the shown case $=2$, no point can be in the
region larger than $2$ and smaller than $4$, i.e. in the upper half of
the distribution.

From the figure it is probably clear that the value of
\inputparameter{collop\_bspline\_dist} should not differ much from one,
as the distributions for $1.5$ and $0.5$ seem already quite extreme.

\paragraph{profile input}
An hdf5 file, which contains the profile (e.g. density, temperature)
data, required for the creation of a radial scan.
This is only used by \neotwo when run to create the folders with the
inputs for a radial scan.
The values from the file will be read, and writen to the \file{neo2.in}
file at the correct location. Interpolation?
