\chapter{Input\label{ch:input}}
This chapter describes the input files of \neotwo and their formats.
Note that this is neither considered to be a reference, nor to give a
detailed description of the formats/input files.\\
The chapter is splited into two sections, the first for the parallel
version, the second for the quasilinear version. Paragraphs within each
section list the input files, that might be necessary, in alphabetical
order by filename/type.

\section{PAR}


\section{QL}


\paragraph{boozer}
Read in subroutine \codevariable{neo\_read} in \file{neo\_sub.f90} (only
the axisymetric part?).

\paragraph{efit}
Efit files are read in \file{field\_divB0} in the routines
\codevariable{read\_dimeq1} (basic size information) and
\codevariable{read\_eqfile1} (all the data, and again the basic size
information).

For efit case, input is assumed to be in SI units. The subroutine
\codevariable{field\_eq} (which calls the reading routines above), will
then convert the values to CGS units.

\paragraph{neo.in}
A fixed format file.

Read in subroutine \codevariable{neo\_read\_control} in
\file{neo\_sub.f90}.

\paragraph{neo2.in}
A namelist file.
Read in main file.

Memory scales at least with cube of lag parameter, as the number of
non-zero elements of the sparse matrix scales with the cube of number of
basis polynoms.

\paragraph{profile input}
An hdf5 file, which contains the profile (e.g. density, temperature)
data, required for the creation of a radial scan.
This is only used by \neotwo when run to create the folders with the
inputs for a radial scan.
The values from the file will be read, and writen to the \file{neo2.in}
file at the correct location. Interpolation?
