\chapter{Output}
This chapter describes the output of the code.
The first section those that is common to both versions of the code, the
second those of the parallel version, and the last section will describe
the output of the quasilinear version.
For each file there is a subsection. The main groups of hdf5 files are
described in paragraphs.

%%%%%%%%%%%%%%%%%%%%%%%%%%%%%%%%%%%%%%%%%%%%%%%%%%%%%%%%%%%%%%%%%%%%%%%%
%%%%%%%%%%%%%%%%%%%%%%%%%%%%%%%%%%%%%%%%%%%%%%%%%%%%%%%%%%%%%%%%%%%%%%%%
\section{Both}

%%%%%%%%%%%%%%%%%%%%%%%%%%%%%%%%%%%%%%%%%%%%%%%%%%%%%%%%%%%%%%%%%%%%%%%%
%%%%%%%%%%%%%%%%%%%%%%%%%%%%%%%%%%%%%%%%%%%%%%%%%%%%%%%%%%%%%%%%%%%%%%%%
\section{Parallel version}

\subsection{amat\_after.dat}
\subsection{amat\_before.dat}

\subsection{bvec\_after.dat}
\subsection{bvec\_before.dat}
\subsection{bvec\_solution.dat}
\subsection{collop.h5}
\subsection{collpar.dat}

\subsection{efinal.h5}

\subsection{eta\_m.dat}
\subsection{eta\_ori.dat}
\subsection{eta\_p.dat}

\subsection{evolve.h5}

\subsection{fin\_dims.dat}
\subsection{fin\_flux\_m.dat}
\subsection{fin\_flux\_p.dat}

\subsection{fin\_source\_m.dat}
\subsection{fin\_source\_p.dat}

\subsection{fulltransp.h5}
This contains the input parameters ``lag'' and ``leg'', which is a point
to remember when comparing files of different runs (especially
convergence tests/scans).

\paragraph{k_cof}
Related to a integration constant in the expression for the parallel
flow. For definition see below Eq. (2.26) in \cite{Martitsch:Thesis:16}.

\subsection{magnetics.h5}
\paragraph{device}
\paragraph{fieldline}
\paragraph{fieldperiod}
\paragraph{fieldpropagator}
\paragraph{fieldripple}
\paragraph{general}
\paragraph{surface}

\subsection{MatrixD\_mmp-gee.dat}
\subsection{MatrixI\_mmp-gee.dat}
\subsection{MatrixNu\_mmp-gee.dat}
\subsection{MatrixNu\_mmp-ginf.dat}

\subsection{neo2\_config.h5}
Contains the settings (from neo2.in) of the run, as well as the hash of
the git version from which the code was build. To support this, also
some of the cmake build variables are included.
This means some of the fields (the /metadata group) should be
excluded when comparing different runs.

The namelists are recreated as groups.

\subsection{phi\_placement\_problem.dat}

\subsection{qfmin1.dat}
\subsection{qfmin2.dat}
\subsection{qfmin3.dat}
\subsection{qfplu1.dat}
\subsection{qfplu2.dat}
\subsection{qfplu3.dat}

\subsection{SourceAa123m\_Cm.dat}

\subsection{taginfo.h5}

%%%%%%%%%%%%%%%%%%%%%%%%%%%%%%%%%%%%%%%%%%%%%%%%%%%%%%%%%%%%%%%%%%%%%%%%
%%%%%%%%%%%%%%%%%%%%%%%%%%%%%%%%%%%%%%%%%%%%%%%%%%%%%%%%%%%%%%%%%%%%%%%%
\section{Quasilinear version}

%%%%%%%%%%%%%%%%%%%%%%%%%%%%%%%%%%%%%%%%%%%%%%%%%%%%%%%%%%%%%%%%%%%%%%%%
\subsection{collop.h5}

%%%%%%%%%%%%%%%%%%%%%%%%%%%%%%%%%%%%%%%%%%%%%%%%%%%%%%%%%%%%%%%%%%%%%%%%
\subsection{eta\_ori.dat}

%%%%%%%%%%%%%%%%%%%%%%%%%%%%%%%%%%%%%%%%%%%%%%%%%%%%%%%%%%%%%%%%%%%%%%%%
\subsection{evolve.dat}

%%%%%%%%%%%%%%%%%%%%%%%%%%%%%%%%%%%%%%%%%%%%%%%%%%%%%%%%%%%%%%%%%%%%%%%%
\subsection{*.sp.chk.dat}

%%%%%%%%%%%%%%%%%%%%%%%%%%%%%%%%%%%%%%%%%%%%%%%%%%%%%%%%%%%%%%%%%%%%%%%%
\subsection{*.sp.cur.dat}

%%%%%%%%%%%%%%%%%%%%%%%%%%%%%%%%%%%%%%%%%%%%%%%%%%%%%%%%%%%%%%%%%%%%%%%%
\subsection{*.sp.dat}

%%%%%%%%%%%%%%%%%%%%%%%%%%%%%%%%%%%%%%%%%%%%%%%%%%%%%%%%%%%%%%%%%%%%%%%%
\subsection{*.sp.log.dat}

%%%%%%%%%%%%%%%%%%%%%%%%%%%%%%%%%%%%%%%%%%%%%%%%%%%%%%%%%%%%%%%%%%%%%%%%
\subsection{MatrixD\_mmp-gee.dat}

%%%%%%%%%%%%%%%%%%%%%%%%%%%%%%%%%%%%%%%%%%%%%%%%%%%%%%%%%%%%%%%%%%%%%%%%
\subsection{MatrixI\_mmp-gee.dat}

%%%%%%%%%%%%%%%%%%%%%%%%%%%%%%%%%%%%%%%%%%%%%%%%%%%%%%%%%%%%%%%%%%%%%%%%
\subsection{MatrixNu\_mmp-gee.dat}

%%%%%%%%%%%%%%%%%%%%%%%%%%%%%%%%%%%%%%%%%%%%%%%%%%%%%%%%%%%%%%%%%%%%%%%%
\subsection{MatrixNu\_mmp-ginf.dat}

%%%%%%%%%%%%%%%%%%%%%%%%%%%%%%%%%%%%%%%%%%%%%%%%%%%%%%%%%%%%%%%%%%%%%%%%
\subsection{neo2\_config.h5}
Note that this file is not written in the multispecies case.

%%%%%%%%%%%%%%%%%%%%%%%%%%%%%%%%%%%%%%%%%%%%%%%%%%%%%%%%%%%%%%%%%%%%%%%%
\subsection{precom\_collop.h5}

\paragraph{Amm}

\paragraph{C\_m}

\paragraph{M\_transform\_}

\paragraph{M\_transform\_inv}

\paragraph{ailmm\_aa}

\paragraph{anumm\_aa}

\paragraph{asource}

\paragraph{denmm\_aa}

\paragraph{meta}
Here the values of some of the input variables are documented.

\paragraph{version}

\paragraph{weightden}

\paragraph{weightdenerg}

\paragraph{weightlag}

\paragraph{weightparflow}

\paragraph{x1mm}

\paragraph{x2mm}
