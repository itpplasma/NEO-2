\chapter{Running}

\section{Simple}
Both variants of the code have hard coded names for parameter-input
files. These must be named ``neo.in'' and ``neo2.in''. The former may
not be needed for all configurations. It is a fixed format file, i.e.
the form and order of parameters is fixed. ``neo2.in'' on the other hand
is a standard fortran namelist file.
The paramete-input files define the file names for the input of the
magnetic equilibrium (background and perturbed).

\subsection{Multispecies}
A speciality of this multispecies run, is that the number of processors
must be equal to the number of species. Other settings might produce
results, but these need not to be correct.
There will be a warning if one attemps to run a multispecies simulation
with the number of processors not equal the number of species, but the
code will continue, as this case might be of interest for testing
things.

\section{Condor}

Create a submit description file.

For more information consult the online manual http://research.cs.wisc.edu/htcondor/manual/.

Commit the jobs using ``condor\_submit''.
You can get the status of your jobs with ``condor\_q''. The option
``--nobatch'' will show you a list single jobs, not summaries for
complete batches.
you can remove jobs from the queue with ``condor\_rm ID'', where ID can
be those of a single run (e.g. 15.12), or those of a whole batch (e.g.
15).
