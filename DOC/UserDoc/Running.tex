\chapter{Running}
This chapter will describe the basic means and tools necessary to run a
\neotwo simulation.
One section will also describe some examples of complete workflows, i.e.
if necessary how to create/process the input, how to do scans and how to
process the data.

%%%%%%%%%%%%%%%%%%%%%%%%%%%%%%%%%%%%%%%%%%%%%%%%%%%%%%%%%%%%%%%%%%%%%%%%
%%%%%%%%%%%%%%%%%%%%%%%%%%%%%%%%%%%%%%%%%%%%%%%%%%%%%%%%%%%%%%%%%%%%%%%%
\section{Computing Ressources}
This section should give a brief overview of computing ressources needed
for runs of \neotwo, and of scalings.

%%%%%%%%%%%%%%%%%%%%%%%%%%%%%%%%%%%%%%%%%%%%%%%%%%%%%%%%%%%%%%%%%%%%%%%%
\subsection{Computing time}

%%%%%%%%%%%%%%%%%%%%%%%%%%%%%%%%%%%%%%%%%%%%%%%%%%%%%%%%%%%%%%%%%%%%%%%%
\subsection{Memory}
Memory scales with q? Code needs one poloidal turn, nsteps
is number of points for one toroidal turn. Thus for a radial scan,
differences of $\tilde \times 4$ are to be expected between core and
edge.

%%%%%%%%%%%%%%%%%%%%%%%%%%%%%%%%%%%%%%%%%%%%%%%%%%%%%%%%%%%%%%%%%%%%%%%%
%%%%%%%%%%%%%%%%%%%%%%%%%%%%%%%%%%%%%%%%%%%%%%%%%%%%%%%%%%%%%%%%%%%%%%%%
\section{Simple}
Both variants of the code have hard coded names for parameter-input
files. These must be named ``neo.in'' and ``neo2.in''. The former may
not be needed for all configurations. It is a fixed format file, i.e.
the form and order of parameters is fixed. ``neo2.in'' on the other hand
is a standard fortran namelist file.
The parameter-input files define the file names for the input of the
magnetic equilibrium (background and perturbed).

%%%%%%%%%%%%%%%%%%%%%%%%%%%%%%%%%%%%%%%%%%%%%%%%%%%%%%%%%%%%%%%%%%%%%%%%
\subsection{Multispecies}
A speciality of this multispecies run, is that the number of processors
must be equal to the number of species. Other settings might produce
results, but these need not to be correct.
There will be a warning if one attemps to run a multispecies simulation
with the number of processors not equal the number of species, but the
code will continue, as this case might be of interest for testing
things.

Note that it is not enough to change the parameter num\_species in the
profile file, as there is also a variable rel\_stages, which stores the
number of species for each flux surface.

The variable num\_species in namelist multi\_spec will be set
automatically, if you use \neotwo to set up the runs for the flux
surfaces.

%%%%%%%%%%%%%%%%%%%%%%%%%%%%%%%%%%%%%%%%%%%%%%%%%%%%%%%%%%%%%%%%%%%%%%%%
%%%%%%%%%%%%%%%%%%%%%%%%%%%%%%%%%%%%%%%%%%%%%%%%%%%%%%%%%%%%%%%%%%%%%%%%
\section{Condor}

Create a submit description file.

For more information consult the online manual http://research.cs.wisc.edu/htcondor/manual/.

Commit the jobs using ``condor\_submit''.
You can get the status of your jobs with ``condor\_q''. The option
``--nobatch'' will show you a list single jobs, not summaries for
complete batches.
you can remove jobs from the queue with ``condor\_rm ID'', where ID can
be those of a single run (e.g. 15.12), or those of a whole batch (e.g.
15).

%%%%%%%%%%%%%%%%%%%%%%%%%%%%%%%%%%%%%%%%%%%%%%%%%%%%%%%%%%%%%%%%%%%%%%%%
%%%%%%%%%%%%%%%%%%%%%%%%%%%%%%%%%%%%%%%%%%%%%%%%%%%%%%%%%%%%%%%%%%%%%%%%
\section{Preparation of Runs and Postprocession}
This section aims to describe the basic steps that are necessary to be
able to run neo-2, in the sense of preparing input files. Which includes
the configuration/parameter files as well as files for magnetic
equilibria and perturbation and the profiles.\\
Note, that this section only aims at describing the most common steps
and most common workflows, not all the steps/workflows, that might
occour.

%%%%%%%%%%%%%%%%%%%%%%%%%%%%%%%%%%%%%%%%%%%%%%%%%%%%%%%%%%%%%%%%%%%%%%%%
\subsection{QL run for the computation of torque/plasma rotation}

\paragraph{Directory setup}
It is recommended to create a folder in your directory on \tpath{/temp/},
which will contain all the runs that can share the same precomputation
run. Subfolder contain then the scans over the radial domain, i.e. there
will be three levels - the base for the scans, the folders for the scans
over rotation velocity and the folders for the radial position. Of
course the last level does not need to be set up manually, \neotwo can
do this.

Preparation of magnetic and profile files might also be done in the top
folder.

\paragraph{Input profile}
\neotwo requires the profile input to be in the form of a hdf5 file. The
matlab function \program{generate\_neo2\_profile.m} is available to create
these from text-data files. As input, loaded from textfile, the function
requires the profiles of density, temperature, rotation velocity as a
function of $\rho_{poloidal}$. Second the rotation veloctiy as a
function of major radius is needed, if the rotation velocity is given in
$m/s$ (not if it is given in $rad/s$). Third, the values of
$\rho_{toroidal}$ corresponding to the values of $\rho_{poloidal}$ are
needed.

Copy the directory with the profiles (e.g. \tpath{/temp/kasilov/ASDEX\_MODES\_IGOSHINE/neo2\_profiles/}).

Run the matlab function \program{generate\_neo2\_profile.m} (does not
work with octave, as octave so far lacks the necessary hdf5 functions).
Easiest way to do so, is to write a small script that sets the required
input parameters and calls the function with these, as this makes it
easier to remember the settings and to change them if necessary.
The first line of the file with the function contains an example how to
run the script as comment.
The section called ``input'' (via comment) contains the parameters that
might need to be changed. The script relies on the function
\program{load\_profile\_data} to load the data (this function works also with
octave).

The output will be an hdf5 file with the profiles necessary to create
simulations for each radial position.

Be carefull with large values for the number of profiles. While the
(matlab) script has no problem with that, and the output hdf5 file
should be fine, the generation of the subfolders by \neotwo can run into
problems. Namely the names of the folders for some fluxsurfaces might
clash due to the naming scheme.
Observed has for example 400 flux surfaces resulting in 399 subfolders
and 1000 flux surfaces into 996 subfolders.

Check that the magnitude of the density and temperature match the
expecations, to verify that the units of the input have been interpreted
correctly. (Density exponent $\approx 12-14$, temperature exponent $\approx (-10) - (-8)$?)

\paragraph{Magnetic equilibrium}
Get as efit or boozer. If you need to create perturbed files, then the
boozer format is required.
There is a fortran program to convert efit into boozer. It is located at
\tpath{/proj/plasma/Neo2/NTV/ASDEX\_MODES\_IGOSHINE/EFIT\_TO\_BOOZER/}.

\paragraph{Magnetic perurbation}
Creating the magnetic perurbation files is done in three steps.

First the displacements need to be calculated. There is the
matlab/octave function \program{create\_displacement} to do this. It
will produce a file with four columns, rho, rho squared, displacement
for kink mode and displacement for tearing mode. Example usage of the
function
\begin{verbatim}
plot_data = false;
file_displacement_268 = 'displacement_morepoints_35568_t2p68.txt';
create_displacement(file_displacement_268,...
  1000, 0.25, 0.44945, 5, [0.06, 1, 1, 1], plot_data);
\end{verbatim}
With the displacement and the background field (in a boozer file), the
perturbed field is calculated. The matlab/octave function
\program{create\_asdex\_perturb} can do this. Example usage of the
function
\begin{verbatim}
file_base = 'eqdsk_35568_2.68800';
file_displacement_268 = 'displacement_morepoints_35568_t2p68.txt';
amplitudes = [1/100, 1/150];
phases = [0, pi];
plot_data = false;
create_asdex_perturb(file_base, file_displacement_268, amplitudes, phases, plot_data)
\end{verbatim}

The script \program{script\_create\_displacements\_and\_perturbation\_35568.m}
located at \tpath{/proj/plasma/Neo2/NTV/ASDEX\_MODES\_IGOSHINE/EQUILIBRIA\_35568/PERTURB\_BOOZER\_FILE}
can serve as an example how to use these two functions to create the
boozer files with perturbations. Note, that this script creates output
for two different time steps, and three different files for each of
these: a file with kink and tearing perturbation, one with only the kink
mode and one with only the tearing mode. The displacement file is the
same for each of these three cases.

If you need three perturbation files, with only kink, only tearing and
both, respectively, then the matlab/octave function
\program{create\_kink\_tearing\_perturbation} can be used to create the
displacement and the three perturbation files at once (not tested yet).

The perturbed magnetic fields can be checked with the help of the
method \program{contours\_in\_r\_z\_plane} of the python class
\program{BoozerFile}. It will write contour lines to a file, that can be
plotted, e.g. with gnuplot.

The class also has method \program{calculate\_island\_width} to
calculate the island width, to aid in
deciding if the settings, e.g. for the amplitudes of the modes, are
suitable.

Third step is then the extraction of the n-modes from the produced
boozer file. Usually only one of the modes is used for simulations.
For the extraction there is the matlab/octave function
\program{extract\_pert\_field}.
Note that the n=0 mode is again the background.


\paragraph{Precomputation Run}
For further information regarding the input files, see Ch. \ref{ch:input}.

Create in the main directory a subdirectory for the precomputation run.
Create links to the background and perturbation file, as well as the
file with the profiles (of density, temperature, etc), these are required
as inputs for \neotwo.
Copy neo.in and neo2.in into the folder, these files contain the input
parameters and also define the names of the input files, so make sure
the entries match the file names. Create or copy a submit file
for condor, if you want to use condor. There is an example file in the
\tpath{DOC} folder. Create a file vphiref.in, containing the line
``vphiref = 0''. This file contains the rotation velocity and is read by
some scripts (for the precomputation it should be zero). Copy the
\neotwo executable into the folder (usually you do not want a debug
version as it is slower). Execute \neotwo, to produce the folders for
the radial scan. Next step is to start the runs for the radial
positions, either manually or via condor with
\begin{verbatim}
condor_submit submitfilename
\end{verbatim}

Remember that the number of processors must be equal to the number of
species.

\paragraph{Computation Run}
For a computation run, make a copy of the whole precomputation
direcetory, i.e. including files, subfolders and files in the
subfolders. Just the condor output, standard (scan.log) and error
(err.*) might/should be removed. Change the value in the file vphiref.in
to the one selected for this run. Note that some scripts requires this
to be not in exponential format (i.e. the value should be written as,
e.g. -100.2). Now some values in all the input files need to be adjusted.
Namely multispec::vphi, collision::lsw\_read\_precom and
collision::lsw\_write\_precom.
Instead of doing this manually, you can use the shell script
\program{set\_neo2in.sh} or the python function \program{set\_neo2in} of
\program{scan\_nonhdf5\_tools.py}.
Start the simulations as you would start a precomputation run.

\paragraph{Checks}

\paragraph{Postprocessing Data}
The torque data is in \file{neo2\_multispecies\_out.h5}, but so far the
data is scattered over the folders for the radial scan. The python
function \program{copy\_hdf5\_from\_subfolders\_to\_single\_file} can
combine the files from the subfolders into a single file. The python
function \program{reshape\_hdf5\_file}, can reshape sucha file, so that
it no longer contains an entry for each subfolder, but instead arrays
over radius. The python function \program{collect\_and\_reshape} does
both at once.

Another aproach is the python function \program{postproc\_torque} from
the file with the same name. This will read the data from different
points of the scan, and for each of them from the radial folders. The
output is the torque over vshift, written to a file.

\paragraph{Additional Data processing}
Here diagnostics are given, that are not related to physics.

There is a python class condor\_log in
\program{scan\_nonhdf5\_tools.py}, for parsing the file with the output
by condor. This can be used to get information about memory consumption
and runtime.

%~ ########################################################################

%~ 13) If you are not sure that precomputation is finished, run
%~ ./check_after_precomp.sh RUN_AUG_template in the root directory and then
%~ condor_submit submit_fixed in the template directory.

%~ Final check
%~ 16) If you are not sure that all computations are finished, run ./check_after.sh in the root directory and then condor_submit submit_fixed in the shifted directory.
%~ You can check maximal memory request by running ./max_condor_memory_all.sh in the root directory.
%~ Postprocession

%~ 18) Data in “NTV_torque_int_all.dat” can be ploted by running plot_NTV_torque_int_all.m Matlab script. This produces 3 plots in EPS and PDF format each, named integral_torque_all_zoom.*, integral_torque_all.*, torque_vs_vphi_shift.*. First two have 3 curves – total, electron and ion torque, the last one has only the total torque in.

%~ 19) If you need further plots and data for different shift (torque density, diffusion coefficients etc.) run in Matlab  ./plot_all_shift.m
%~ All plots and data will be stored in subdirectory plots/ and export_dat/ resepectively
%~ 20) If you need further plots or data export (e.g. for Clemente) make a copy of /temp/aradi_m/neo2_plotter/
%~ Run neo2_plotter.m. All data will be exported to data folder. You can modify datasets and plot export path in the file.
