\chapter{Theory}
This chapter is supposed to give a brief overview of the most important
aspects of the theory.

%%%%%%%%%%%%%%%%%%%%%%%%%%%%%%%%%%%%%%%%%%%%%%%%%%%%%%%%%%%%%%%%%%%%%%%%
%%%%%%%%%%%%%%%%%%%%%%%%%%%%%%%%%%%%%%%%%%%%%%%%%%%%%%%%%%%%%%%%%%%%%%%%
\section{Placements of eta levels}

\begin{equation}
  \eta = \frac{1 - \lambda^2}{B}
\end{equation}
with pitch angle parameter
\begin{equation}
  \lambda = \frac{v_{\parallel}}{v}
\end{equation}

As there is no dependence of $\eta$ on sign of velocity, there are two
distribution functions, $f^{+}$ and $f^{-}$ which are linked over
boundary condition at
\begin{equation}
  \eta_B = \frac{1}{B}
\end{equation}

Depends on input parameters, e.g. \codevariable{binsplit::eta\_part},
\codevariable{settings::eta\_part\_global},
\codevariable{settings::eta\_part\_globalfac},
\codevariable{settings::eta\_part\_globalfac\_p},
\codevariable{settings::eta\_part\_globalfac\_t},
\codevariable{settings::eta\_part\_trapped}.

If a field line closes after only a few rotations, this can lead to bad
sampling of the flux surface. The code tries to avoid this by a small
shift in the eta level, if such a case is found.\cite{Kapper:Diss:17}[Ch. 3.2]

%%%%%%%%%%%%%%%%%%%%%%%%%%%%%%%%%%%%%%%%%%%%%%%%%%%%%%%%%%%%%%%%%%%%%%%%
%%%%%%%%%%%%%%%%%%%%%%%%%%%%%%%%%%%%%%%%%%%%%%%%%%%%%%%%%%%%%%%%%%%%%%%%
\section{Regularization}
The system of equation for the symmetric background is singular. It
can be solved nevertheless, but this results in a ``random'' offset.

Currently this is solve be regularization of the equations, i.e. a
slight change to the equation system, to make it non-singular.
\codevariable{isw\_regper}

The code parameters \codevariable{epserr\_sink} and
\codevariable{epserr\_sink\_cmplx} belong to old variant of
regularization.

%%%%%%%%%%%%%%%%%%%%%%%%%%%%%%%%%%%%%%%%%%%%%%%%%%%%%%%%%%%%%%%%%%%%%%%%
%%%%%%%%%%%%%%%%%%%%%%%%%%%%%%%%%%%%%%%%%%%%%%%%%%%%%%%%%%%%%%%%%%%%%%%%
\section{Calculation of Torque}
The code calculates the torque for each species using the formula
\begin{equation}
       T_{\phi}^{NA} = &
            fac\_flux\_force Z e \Gamma^{NA}
\end{equation}
where
\begin{equation}
  fac\_flux\_force = -\sqrt{g} B^{\theta} / c,
\end{equation}
and $\Gamma^{NA}$ is the particle flux.
Calculation is only done by the main thread, as
\subroutine{ntv\_mod::write\_multispec\_output} is only called by main
thread.

%%%%%%%%%%%%%%%%%%%%%%%%%%%%%%%%%%%%%%%%%%%%%%%%%%%%%%%%%%%%%%%%%%%%%%%%
%%%%%%%%%%%%%%%%%%%%%%%%%%%%%%%%%%%%%%%%%%%%%%%%%%%%%%%%%%%%%%%%%%%%%%%%
\section{Rotation}

For slow rotation, plasma is incompressible, i.e. $\nabla \ve{v} = 0$
holds. This is can be used to determine the parallel part of the
velocity from the perpendicular part. This in turn is calculated from
the crossproduct of the ideal MHD momentum conservation equation with
the magnetic field.
(In the code the electric field is computed from the rotation?)

Depends on input paramters, e.g. \codevariable{settings::vphi}.

%%%%%%%%%%%%%%%%%%%%%%%%%%%%%%%%%%%%%%%%%%%%%%%%%%%%%%%%%%%%%%%%%%%%%%%%
%%%%%%%%%%%%%%%%%%%%%%%%%%%%%%%%%%%%%%%%%%%%%%%%%%%%%%%%%%%%%%%%%%%%%%%%
\section{Momentum conservation in the precomputed matrices}
\begin{equation}
  R_{\alpha\alpha'} = \int d^3p m_{\alpha} v_{\parallel} St(f_{\alpha}, f_{\alpha'})
\end{equation}
Due to newtwon
\begin{equation}
  R_{\alpha\alpha'} + R_{\alpha'\alpha}
\end{equation}
or
\begin{equation}
  R_{\alpha\alpha'} = - R_{\alpha'\alpha},
\end{equation}
i.e. $R_{\alpha\alpha'}$ is antisymmetric.

Linearize the collision operator
\begin{equation}
    \int d^3p m_{\alpha} v_{\parallel} St(\delta f_{\alpha}, f_{0,\alpha'})
  + \int d^3p m_{\alpha} v_{\parallel} St(f_{0,\alpha}, \delta f_{\alpha'})
  + \int d^3p m_{\alpha'} v_{\parallel} St(\delta f_{\alpha'}, f_{0, \alpha})
  + \int d^3p m_{\alpha'} v_{\parallel} St(f_{0,\alpha'}, \delta f_{\alpha}) = 0
\end{equation}
This must hold for all $\delta f_{\alpha'}$ so also for $\delta f_{\alpha'} = 0$.
In this case
\begin{equation}
    \int d^3p m_{\alpha} v_{\parallel} St(\delta f_{\alpha}, f_{0,\alpha'})
  + \int d^3p m_{\alpha'} v_{\parallel} St(f_{0,\alpha'}, \delta f_{\alpha}) = 0
\end{equation}
This means
\begin{equation}
    \int d^3p m_{\alpha} v_{\parallel} L_D^{\alpha\alpha'} \delta f_{\alpha}
  + \int d^3p m_{\alpha'} v_{\parallel} L_I^{\alpha'\alpha} \delta f_{\alpha} = 0
\end{equation}
Now change integration from $d^3p$ to $d^3v$, and change the distribution
function accordingly (to absorb the resulting factors).
\begin{equation}
    \int d^3v m_{\alpha} v_{\parallel} L_D^{\alpha\alpha'} \delta f_{\alpha}
  + \int d^3v m_{\alpha'} v_{\parallel} L_I^{\alpha'\alpha} \delta f_{\alpha} = 0
\end{equation}
In a second step change make the velocity coordinates specific
\begin{equation}
  \int d^3v = 2\pi \int\limits_{0}^{\infty} dv v^2 \int\limits_{-1}^{+1} d\lambda
\end{equation}
which results in
\begin{equation}
    2\pi \int\limits_{0}^{\infty} dv v^2 \int\limits_{-1}^{+1} d\lambda m_{\alpha} v_{\parallel} L_D^{\alpha\alpha'} \delta f_{\alpha}
  + 2\pi \int\limits_{0}^{\infty} dv v^2 \int\limits_{-1}^{+1} d\lambda m_{\alpha'} v_{\parallel} L_I^{\alpha'\alpha} \delta f_{\alpha} = 0
\end{equation}
Concentrate for now on the first term
\begin{equation}
  R_D = \frac{1}{2\pi} 2\pi \int\limits_{0}^{\infty} dv v^2 \int\limits_{-1}^{+1} d\lambda m_{\alpha} v_{\parallel} L_D^{\alpha\alpha'} \delta f_{\alpha}
\end{equation}
and introduce $\delta f_{\alpha} (v_{\parallel}, \lambda) = f_{0\alpha}(v_{\parallel}) g(\lambda)$
\begin{equation}
  R_D = \int\limits_{0}^{\infty} dv v^2 \int\limits_{-1}^{+1} d\lambda m_{\alpha} v_{\parallel} L_D^{\alpha\alpha'} \delta f_{\alpha}
\end{equation}

%%%%%%%%%%%%%%%%%%%%%%%%%%%%%%%%%%%%%%%%%%%%%%%%%%%%%%%%%%%%%%%%%%%%%%%%
%%%%%%%%%%%%%%%%%%%%%%%%%%%%%%%%%%%%%%%%%%%%%%%%%%%%%%%%%%%%%%%%%%%%%%%%
\section{Relations of axisymmetric transport coefficients for multispecies plasmas}
Ambipolarity of the radial fluxes means
\begin{equation}
  \sum_{\alpha} \Gamma^{\alpha} = 0.
\end{equation}
Expressing the flux via the flux-force relations, we have
\begin{equation}
  0 = - \sum_{\alpha} e_{\alpha} n_{\alpha} \sum_{\alpha'} \left( D_{11}^{\alpha\alpha'} A_{1}^{\alpha'} + D_{12}^{\alpha\alpha'} A_{2}^{\alpha'} + D_{13}^{\alpha\alpha'} A_{3}^{\alpha'}\right).
\end{equation}
Inserting the thermodynamic forces
\begin{equation}
  0 = - \sum_{\alpha} \sum_{\alpha'} e_{\alpha} n_{\alpha} \left( D_{11}^{\alpha\alpha'} (\frac{1}{n_{\alpha'}} \pd{n_{\alpha'}}{r} - \frac{e_{\alpha'} E_r}{T_{\alpha'}} - \frac{3}{2 T_{\alpha'}} \pd{T_{\alpha'}}{r})
    + D_{12}^{\alpha\alpha'} \frac{1}{T_{\alpha'}} \pd{T_{\alpha'}}{r} + D_{13}^{\alpha\alpha'} \frac{e_{\alpha'} \langle E_{\parallel} B \rangle}{T_{\alpha'} \langle B^2 \rangle} \right).
\end{equation}
The terms with the derivative of the temperature can be combined. Doing
so, results in
\begin{equation}
  0 = - \sum_{\alpha, \alpha'} e_{\alpha} n_{\alpha} D_{11}^{\alpha\alpha'} \frac{1}{n_{\alpha'}} \pd{n_{\alpha'}}{r}
      + E_r \sum_{\alpha, \alpha'} \frac{e_{\alpha} n_{\alpha} D_{11}^{\alpha\alpha'} e_{\alpha'}}{T_{\alpha'}}
      + \sum_{\alpha, \alpha'} e_{\alpha} n_{\alpha} (\frac{3}{2} D_{11}^{\alpha\alpha'} - D_{12}^{\alpha\alpha'}) \frac{1}{T_{\alpha'}} \pd{T_{\alpha}}{r}
      - \frac{\langle E_{\parallel} B \rangle}{\langle B^2 \rangle} \sum_{\alpha, \alpha'} e_{\alpha} n_{\alpha} D_{13}^{\alpha\alpha'} \frac{e_{\alpha'}}{T_{\alpha'}}.
\end{equation}
It should be safe to assume that these terms can change individually,
which means each of these must be zero on its own
\begin{equation}
  0 = \sum_{\alpha, \alpha'} e_{\alpha} n_{\alpha} D_{11}^{\alpha\alpha'} \frac{1}{n_{\alpha'}} \pd{n_{\alpha'}}{r}
\end{equation}
\begin{equation}
  0 = \sum_{\alpha, \alpha'} \frac{e_{\alpha} n_{\alpha} D_{11}^{\alpha\alpha'} e_{\alpha'}}{T_{\alpha'}}\label{eq_ambipolarity_condition_from_radial_electric_field}
\end{equation}
\begin{equation}
  0 = \sum_{\alpha, \alpha'} e_{\alpha} n_{\alpha} (\frac{3}{2} D_{11}^{\alpha\alpha'} - D_{12}^{\alpha\alpha'}) \frac{1}{T_{\alpha'}} \pd{T_{\alpha}}{r}
\end{equation}
\begin{equation}
  0 = \sum_{\alpha, \alpha'} e_{\alpha} n_{\alpha} D_{13}^{\alpha\alpha'} \frac{e_{\alpha'}}{T_{\alpha'}}
\end{equation}
Another condition can be derived under the assumption that centrifugal
forces are small. Then the corotating frame can be seen as an inertial
system, where the radial flux has to be the same as in the original
system. This is because it is perpendicular to the motion. But in the
corotation frame the radial electric field vanishes, and so the radial
particle flux must be independent of the radial electric field,
\begin{equation}
  \pd{\Gamma^{\alpha}}{E_r} = 0.
\end{equation}
As only the first thermodynamic force depends on the radial electric
field, only those needs to be considered
\begin{equation}
  0 = - n_{\alpha} \pd{}{E_r} \sum_{\alpha'} D_{11}^{\alpha\alpha'} A_1^{\alpha'}.
\end{equation}
Inserting the force
\begin{equation}
  0 = - n_{\alpha} \pd{}{E_r} \sum_{\alpha'} D_{11}^{\alpha\alpha'} \left( \frac{1}{n_{\alpha'}} \pd{n_{\alpha'}}{r} - \frac{e_{\alpha'} E_r}{T_{\alpha'}} - \frac{3}{2 T_{\alpha'}} \pd{T_{\alpha'}}{r} \right),
\end{equation}
and doing the derivative
\begin{equation}
  0 = n_{\alpha} \sum_{\alpha'} D_{11}^{\alpha\alpha'} \frac{e_{\alpha'}}{T_{\alpha'}}.
\end{equation}
This means
\begin{equation}
  0 = \sum_{\alpha'} D_{11}^{\alpha\alpha'} \frac{e_{\alpha'}}{T_{\alpha'}}.
\end{equation}
This is a stricter version of
Eq.~\eqref{eq_ambipolarity_condition_from_radial_electric_field}, as is
easy to see.

%%%%%%%%%%%%%%%%%%%%%%%%%%%%%%%%%%%%%%%%%%%%%%%%%%%%%%%%%%%%%%%%%%%%%%%%
%%%%%%%%%%%%%%%%%%%%%%%%%%%%%%%%%%%%%%%%%%%%%%%%%%%%%%%%%%%%%%%%%%%%%%%%
\section{Incompressible displacements of magnetic field}

Infinitesimal incompressible perpendicular displacement $\ve{\xi}$ must
satisfy conditions
\begin{equation}
  0 = \ve{B} \cdot \ve{xi}
\end{equation}
\begin{equation}
  0 = \nabla \cdot \ve{xi},
\end{equation}
with the magnetic field $\ve{B}$. Expressing this in Boozer coordinates
$(\tho, \theta, \phi)$, where $\rho = \rho_{tor} = \sqrt{s} = \sqrt{\psi_{tor}/\psi_{tor}^a}$
is the normalized radius, we get
\begin{equation}
  0 = B_{\rho} \xi^{\rho} + B_{\theta} \xi^{\theta} + B_{\phi} \xi^{\phi}\label{eq:perpendiculardisplacementboozer}
\end{equation}
\begin{equation}
  0 = \pder{}{\rho} \sqrt{g} \xi^{\rho} + \pder{}{\theta} \sqrt{g} \xi^{\theta} + \pder{}{\phi} \sqrt{g} \xi^{\phi}
  \label{eq:imcompressibledisplacementboozer}
\end{equation}
with covariant magnetic field components $B_{\rho} = B_{\rho}(\rho, \theta)$,
$B_{\theta} = B_{\theta}(\rho}$ and $B_{\phi} = B_{\phi}(\phi)}$. Also
the metric determinant
\begin{equation}
  \sqrt{g} = 2 \psi_{tor}^a \rho \frac{\iota B_{\theta} + B_{\phi}}{B^2}
  \label{eq:metricdeterminant}
\end{equation}
has been used, with $iota = 1/q(rho)$.
We will now assume that $\xi^{\rho}$ is a known function, e.g. from some
analytical model, and we will use Eq.~\eqref{eq:perpendiculardisplacementboozer}
and \eqref{eq:incompressibledisplacementboozer} to determine the other
two components.
To have an estimate of the order of magnitude, remember
$B_{\rho} / B_{\phi} \tilde \beta q / A \ll 1$ and
$B_{\theta} / B_{\phi} \tilde 1/(qA^2) \ll 1$ where $\beta$ is plasma
beta, and $A$ is aspect ratio. With this we get estimates
\begin{equation}
  \xi^{\theta} \tilde \frac{\xi^{\rho}}{m \Delta \rho}
\end{equation}
and
\begin{equation}
  \xi^{\phi} \tilde \frac{max(1, m \Delta \rho q^2 A \beta) \xi^{\theta}}{q A^2} \ll \xi^{\theta}
\end{equation}
with mode number $m$ and mode width $\Delta \rho$. As $\xi^{\phi}$ is
small, we ignore it. Ignoring also variation of $B$ within flux surfaces
(terms of order $1/A$ in Eq.~\eqref{eq:metricdeterminant}), we obtain
an equation for $\xi^{\theta}$
\begin{equation}
  \frac{1}{\rho} \pder{}{\rho}\rho \xi^{\rho} + \pder{\xi^{\theta}}{\theta} = 0.
  \label{eq:equationxitheta}
\end{equation}
Our specific perturbation $\xi^{\rho}$ is of the form
\begin{equation}
  \xi^{\rho} = \xi_K^{\rho} (\rho) \cos (m_K \theta - n \phi) + \xi_T^{\rho} (\rho) \cos (m_T \theta - n \phi) = \xi^{\rho} (\rho, \theta, \phi),
  \label{eq:perturbationkinktearing}
\end{equation}
where the subscripts $K$ and $T$ stand for kink and tearing,
respectively.
The corresponding perurbation $\xi^{\theta}$ is of the form
\begin{equation}
  \xi^{\theta} = - \frac{1}{m_k \rho} \pder{\rho \xi_K^{\rho}}{\rho} \sin (m_K \theta - n \phi)
                 - \frac{1}{m_T \rho} \pder{\rho \xi_T^{\rho}}{\rho} \sin (m_T \theta - n \phi).
\end{equation}
Because the coordinate system $(s, \theta, \phi)$ is more convenient we
change to this one.
The condition Eq.~\eqref{eq:equationxitheta} changes to
\begin{equation}
  \pder{\xi^{s}}{\rho} + \pder{\xi^{\theta}}{\theta} = 0.
\end{equation}
Our infinitesimal displacement Eq.~\eqref{eq:perturbationkinktearing}
can in this coordinates be written as
\begin{equation}
  \xi^{s} = \xi_K^{s} (\rho) \cos (m_K \theta - n \phi) + \xi_T^{s} (\rho) \cos (m_T \theta - n \phi) = \xi^{s} (\rho, \theta, \phi),
  \label{eq:infinitesimaldisplacements}
\end{equation}
with
\begin{equation}
  \xi^{s} = 2 \rho \xi^{\rho} + o(\xi^{\rho}),
\end{equation}
while the respective displacement $\xi^{\theta}$ can be written as
\begin{equation}
  \xi^{\theta} = - \frac{1}{m_K}\pder{\xi_K^{s}}{s} \sin (m_K \theta - n \phi) - \frac{1}{m_T}\pder{\xi_T^{s}}{s} \sin (m_T \theta - n \phi)
\end{equation}

%%%%%%%%%%%%%%%%%%%%%%%%%%%%%%%%%%%%%%%%%%%%%%%%%%%%%%%%%%%%%%%%%%%%%%%%
\subsection{Generalization to finite displacements}
Within approximations of the previous section, where $\xi^{\phi} = 0$
and $B = const$, an incompressible deformation of space corresponds to
an area preserving map $(s, \theta) \rightarrow (s', \theta')$, which
has unit Jacobian
\begin{equation}
  \pder{(s', \theta)'}{(s, \theta)} = 1.
  \label{eq:codnitionunitejacobian}
\end{equation}
In case of infinitesimal displacement the perturbed position is
$s' = s + \xi^{s}$ and $\theta' = \theta + \xi^{\theta}$. For the case
of finite displacements we set
\begin{equation}
  s' = s + \xi^{s}(s, \theta', phi)
\end{equation}
where $\xi^{s}(s, \theta, \phi$ is given by
Eq.~\eqref{eq:infinitesimaldisplacements}. Variable $\theta'$ can be
found from implicit form Eq.~\eqref{eq:codnitionunitejacobian}, which
determines $\theta = \theta(s, \theta', \phi)$.
Namely, using
\begin{equation}
  \pder{(s', \theta')}{(s,\theta)} = \pder{(s',\theta')}{(s, \theta')} \pder{(s, \theta'}{(s, \theta)}
  = \pder{s'}{s} \left( \pder{\theta}{\theta'} \right)^{-1} = 1,
\end{equation}
we get an equation for $\theta$
\begin{equation}
  \pder{}{\theta'} \theta(s, \theta', \phi) = 1 + \pder{}{s} \xi^{s} (s, \theta', \phi)
\end{equation}
Inserting Eq.~\eqref{eq:infinitesimaldisplacements} leads to
\begin{equation}
  \pder{\theta}{\theta'} = 1 + \pder{\xi_K^{s}}{s}\cos(m_K \theta' - n \phi)
                             + \pder{\xi_T^{s}}{s}\cos(m_T \theta' - n \phi)
\end{equation}
with solution
\begin{equation}
  \theta = \theta' + \frac{1}{m_K} \pder{\xi_K^{s}}{s}\sin(m_K \theta' - n \phi)
                   + \frac{1}{m_T} \pder{\xi_T^{s}}{s}\sin(m_T \theta' - n \phi)
\end{equation}
%%%%%%%%%%%%%%%%%%%%%%%%%%%%%%%%%%%%%%%%%%%%%%%%%%%%%%%%%%%%%%%%%%%%%%%%
