\chapter{Notes}
This is intended for scribbling down notes.
Idealy these would be added to an appropriate part of the documentation
at some point.

\section{Mode number n}
The mode number n given in the boozer file has to be multiplied with
\codevariable{nper} to get the actual mode number.

\section{Computations}
That the number of bad modes increases to the axis is to be expected/is observed behaviour.

\section{Indexing of \codevariable{source\_vector}}
The array \codevariable{source\_vector} is set in
\subroutine{source\_flux} (e.g. in \subroutine{ripple\_solver\_ArnoldiOrder2}.
Elements are set according to
\begin{verbatim}
source\_vector(k+1:k+npassing+1) += ...
\end{verbatim}
in loops over \codevariable{istep} and \codevariable{m} (\inputparameter{lag} number).
Note that \codevariable{npassing} is an array over \codevariable{istep}.
Index k is set as
\begin{verbatim}
k = ind_start(istep)+2*(npassing+1)*m
\end{verbatim}
which means that there are \codevariable{npassing} unused points inbeween.

There is some more indexing, which may hide the fact that there is no
unused part.

Supported by the fact that output of whole array shows no unused data.

\section{Use of \codevariable{ispec} in parallel PAR runs}
In collision\_operator\_mems.f90, variable \codevariable{ispec} is set
as thread number, why does this not lead to problems in runs of the PAR
version that use multiple processors? In those runs parallelization is
not done over species.
