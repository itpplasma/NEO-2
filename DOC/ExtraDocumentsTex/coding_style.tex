\documentclass{article}
\usepackage{graphicx}
\usepackage{wrapfig}
\usepackage{amsmath}

\usepackage[english]{babel}
\usepackage[utf8]{inputenc}
\usepackage[T1]{fontenc}

\usepackage[a4paper]{geometry}
\usepackage{authblk}
\usepackage{setspace}
\usepackage{csquotes}
\usepackage{booktabs}
%%%%%%%%%%%%%%%%%% My preamble %%%%%%%%%%%%%%%%%%%%%%%%%%
%\usepackage[numbers, sort&compress]{natbib}
%\setcitestyle{numbers,square,comma,aysep={,},yysep={,},notesep={,}}
%\bibliographystyle{unsrtnat}
%\bibliographystyle{IEEEtranSN}

%References
\usepackage[
    bibstyle=phys,
    biblabel=brackets,
    citestyle=numeric-comp,
    isbn=false,
    doi=false,
    sorting=none,
    url=false,
    defernumbers=true,
    bibencoding=utf8,
    backend=biber,
    %maxbibnames=3,maxcitenames=3,
    %minnames=3,
    maxnames=30,
    ]{biblatex}
\setcounter{biburllcpenalty}{7000}
\setcounter{biburlucpenalty}{8000}
\addbibresource[]{eccd_ref.bib}
\DeclareBibliographyCategory{fullcited}
\newcommand{\mybibexclude}[1]{\addtocategory{fullcited}{#1}}
\DeclareFieldFormat{titlecase}{\MakeCapital{#1}}
\DeclareFieldFormat{sentencecase}{\MakeSentenceCase{#1}}

\usepackage[font=small,labelfont=bf]{caption}
\usepackage{amssymb}
\usepackage{bm}

\title{\textbf{NEO-2 Coding Style\\November 2018}}

\begin{document}

\maketitle

%%%%%%%%%%%%%%%%%%%%%%%%%%%%%%%%%%%%%%%%%%%%%%%%%%%%%%%%%%%%%%%%%%%%%%%%
%%%%%%%%%%%%%%%%%%%%%%%%%%%%%%%%%%%%%%%%%%%%%%%%%%%%%%%%%%%%%%%%%%%%%%%%
\section{Purpose of this document}
The purpose of this document is to provide a style guide for writing new
code and editing existing code of NEO-2 and related scripts and
programs.

It is clear that the code does not follow this standard everywhere (or
maybe not even anywhere), but to say this again, this is intended for
future code and rewriting/editing of existing code.

Usually things are said only once, so check all relevant sections.
Note also that some things are only relevant to a certain tool/language
(e.g. the git section) while others should be followed in general (e.g.
the section about whitespace).

%%%%%%%%%%%%%%%%%%%%%%%%%%%%%%%%%%%%%%%%%%%%%%%%%%%%%%%%%%%%%%%%%%%%%%%%
%%%%%%%%%%%%%%%%%%%%%%%%%%%%%%%%%%%%%%%%%%%%%%%%%%%%%%%%%%%%%%%%%%%%%%%%
\section{code}

Use fortran 2008 standard, with files in free form.

Use 'F90' as extension for files, that require preprocessing and 'f90'
for files which do not require preprocessing.

In all modules/functions/subroutines use ``implicit none''.

Use only what is needed, i.e. use ``use module, only :''.

Use ``private'' as default acess for modules, i.e. only what is
explicitly declared as public can be acessed from outside the module.

For classes ``protected'' should be the default acess for methods
(subroutines/functions of the class) and ``private'' for members
(variables).

The code should compile without warnings.

Put new files in the COMMON directory, unless the new code is definitly
specific to one of the two programs.
If you write something for one code and it might be also used in the
other, then put it into COMMON. Make sure that your code is general
enough to easily use it also from the other program, even if you do not
implement the code for this.

%%%%%%%%%%%%%%%%%%%%%%%%%%%%%%%%%%%%%%%%%%%%%%%%%%%%%%%%%%%%%%%%%%%%%%%%
\subsection{naming}
Use speaking, descriptive names, make them as long as needed but not
longer than needed (and not longer than allowed by the standard).

%%%%%%%%%%%%%%%%%%%%%%%%%%%%%%%%%%%%%%%%%%%%%%%%%%%%%%%%%%%%%%%%%%%%%%%%
\subsection{comments}
(Note: this subsection assumes, doxygen is used for automatic generation
of documentation. If we decide on something different, then this should
be adapted.)\\
For documentation of a variable on the same line use '!<'.
Documentation for subroutines/classes/modules etc., should be put in
front of the subroutine/class/module etc., and be started with '!>' and
continued on following lines with '!>'.

%%%%%%%%%%%%%%%%%%%%%%%%%%%%%%%%%%%%%%%%%%%%%%%%%%%%%%%%%%%%%%%%%%%%%%%%
\subsection{case}
While fortran is not case sensitive, write code as if it is.

Use lowercase for commands.

%%%%%%%%%%%%%%%%%%%%%%%%%%%%%%%%%%%%%%%%%%%%%%%%%%%%%%%%%%%%%%%%%%%%%%%%
\subsection{whitespace}
No tabs
intendation is two whitespaces for every level
contains on same level as module/subroutine/function statement
new level due to
- module
- function/subroutine
- if/elseif/else
- do/do while
- case (each case is on same level as select)

whitespace after if and before then, i.e. ``if () then''

whitespace after end, e.g. ``end do'', ``end if''

%%%%%%%%%%%%%%%%%%%%%%%%%%%%%%%%%%%%%%%%%%%%%%%%%%%%%%%%%%%%%%%%%%%%%%%%
\subsection{Line length}
Lines should be limited to a maximum of 72 characters, including the
whitespace at the beginning of the line.

%%%%%%%%%%%%%%%%%%%%%%%%%%%%%%%%%%%%%%%%%%%%%%%%%%%%%%%%%%%%%%%%%%%%%%%%
\subsection{Line continuation}
Put the continuation sign at the end of the line and as first
non-whitespace character of the next line.

%%%%%%%%%%%%%%%%%%%%%%%%%%%%%%%%%%%%%%%%%%%%%%%%%%%%%%%%%%%%%%%%%%%%%%%%
%%%%%%%%%%%%%%%%%%%%%%%%%%%%%%%%%%%%%%%%%%%%%%%%%%%%%%%%%%%%%%%%%%%%%%%%
\section{git}

Make small but complete commits, containing only one change, i.e. adding
a subroutine (maybe with changes for calling it) or fixing one bug.

After each commit to the main branch testcases should pass.

After a push testcases must pass.

Use branches for developments that will take more than two or three
commits over the course of a day or two.

set user.name and user.email correctly.

example
[user]
  name = Max Musterman
  email = musterman@tugraz.at

Suggested (required?) to set
set push.default = simple
set pull.rebase = preserve

Suggested to activate the standard pre-commit hook. To do so simply
rename the file ``.git/hooks/pre-commit.sample'' to
``./git/hooks/pre-commit''.

%%%%%%%%%%%%%%%%%%%%%%%%%%%%%%%%%%%%%%%%%%%%%%%%%%%%%%%%%%%%%%%%%%%%%%%%
\subsection{tag}

%%%%%%%%%%%%%%%%%%%%%%%%%%%%%%%%%%%%%%%%%%%%%%%%%%%%%%%%%%%%%%%%%%%%%%%%
%%%%%%%%%%%%%%%%%%%%%%%%%%%%%%%%%%%%%%%%%%%%%%%%%%%%%%%%%%%%%%%%%%%%%%%%
\section{(Python or Shell) Scripts}
Scripts that are either requied by neo2 itself, the make-system or the
test system should be put in the corresponding folder of neo2 and added
to the repositiory.

Treat them as code in terms of coding style.

Make them executable, and in the case of python scripts, allow to run
them as main program.

See that the code is reusable, i.e. if a function is required in
multiple scripts, see to put it into an own file which is
included/imported.

\end{document}
