%\documentclass[12pt]{article}
%\documentclass[preprint,prb,aps]{revtex4}
\documentclass[preprint,prb,aps]{revtex4-1}
%\documentclass[preprint,prb,aps]{revtex}
%\sloppy
\textheight24.0cm
\textwidth18.0cm
\topmargin-1.5cm
\oddsidemargin-1.0cm
\evensidemargin0.0cm
\usepackage{times}
\usepackage{amsmath}
\DeclareMathOperator{\sign}{sign}
\usepackage{exscale}
%\usepackage{epsfig}
\usepackage{graphicx}

\usepackage{color}
\newcommand{\red}[1]{{\color{red} #1}}
\newcommand{\green}[1]{{\color{green} #1}}
\newcommand{\blue}[1]{{\color{blue} #1}}

%\newcommand{\baselinestretch}{2}
\newcommand{\be}[1]{\begin{equation} \label{#1}}
\newcommand{\ee}{\end{equation}}
\newcommand{\bea}[1]{\begin{eqnarray} \label{#1}}
\newcommand{\eea}{\end{eqnarray}}
\newcommand{\bean}{\begin{eqnarray*}}
\newcommand{\eean}{\end{eqnarray*}}

\newcommand{\non}{\nonumber\\}
\newcommand{\eq}[1]{(\ref{#1})}
\newcommand{\difp}[2]{\frac{\partial #1}{\partial #2}}
\newcommand{\br}{{\bf r}}
\newcommand{\bR}{{\bf R}}
\newcommand{\bA}{{\bf A}}
\newcommand{\bB}{{\bf B}}
\newcommand{\bE}{{\bf E}}
%\newcommand{\bm}{{\bf m}}
\renewcommand{\bm}{{\bf m}}
\newcommand{\bn}{{\bf n}}
\newcommand{\bN}{{\bf N}}
\newcommand{\bp}{{\bf p}}
\newcommand{\bP}{{\bf P}}
\newcommand{\bF}{{\bf F}}
\newcommand{\by}{{\bf y}}
\newcommand{\bz}{{\bf z}}
\newcommand{\bZ}{{\bf Z}}
\newcommand{\bV}{{\bf V}}
\newcommand{\bv}{{\bf v}}
\newcommand{\bu}{{\bf u}}
\newcommand{\bx}{{\bf x}}
\newcommand{\bX}{{\bf X}}
\newcommand{\bW}{{\bf W}}
\newcommand{\bJ}{{\bf J}}
\newcommand{\bj}{{\bf j}}
\newcommand{\bk}{{\bf k}}
\newcommand{\bTheta}{{\bf \Theta}}
\newcommand{\btheta}{{\boldsymbol\theta}}
\newcommand{\bOmega}{{\bf \Omega}}
\newcommand{\bomega}{{\boldsymbol\omega}}
\newcommand{\brho}{{\boldsymbol\rho}}
\newcommand{\rd}{{\rm d}}
\newcommand{\rJ}{{\rm J}}
\newcommand{\ph}{{\varphi}}
\newcommand{\te}{\theta}
\newcommand{\tht}{\vartheta}
\newcommand{\vpar}{v_\parallel}
\newcommand{\vparkb}{v_{\parallel k b}}
\newcommand{\vparkm}{v_{\parallel k m}}
\newcommand{\Jpar}{J_\parallel}
\newcommand{\ppar}{p_\parallel}
\newcommand{\Bpstar}{B_\parallel^*}
\newcommand{\intpi}{\int\limits_{0}^{2\pi}}
\newcommand{\summ}{\sum \limits_{m=-\infty}^\infty}
\newcommand{\tb}{\tau_b(\uv)}
\newcommand{\bh}{{\bf h}}
\newcommand{\cE}{{\cal E}}
\newcommand{\bsigma}{{\boldsymbol\sigma}}
\newcommand{\bS}{{\mathbf S}}
\newcommand{\bI}{{\mathbf I}}
\newcommand{\odtwo}[2]{\frac{\rd #1}{\rd #2}}
\newcommand{\pdone}[1]{\frac{\partial}{\partial #1}}
\newcommand{\pdtwo}[2]{\frac{\partial #1}{\partial #2}}
\newcommand{\ds}{\displaystyle}
%\newcommand{\bc}{\begin{center}
%\newcommand{\ec}{\end{center}}
%\input{prespan}

\begin{document}

\section{Solution vector}
Solution is obtained after a call
\\
{\bf
3277     CALL solve\_eqs(.TRUE.)
}
\\
and is contained in the inout variable
{\bf
source\_vector\_all(:,1:3,ispecp)
}
where
\\
{\bf :} $= K$ is the phase space discretization index
\\
{\bf 1:3} $= L$ is a thermodynamic force index
\\
{\bf ispecp} $= \alpha^\prime$ is the driving species index
\\
We consider a single species case, $\alpha^\prime=\alpha$ and skip the species index.
Phase space discretization index combines 3 indices, $K => (i,k,m)$, where
$i$ numbers $\varphi$ steps, $k$ numbers $\eta$ levels and $m$ numbers base functions.
\\
We denote the value of the inout variable on entry as
\\
{\bf source\_vector\_all(:,1:3,ispecp)}$|_{entry}= B_{KL}=b_{(i,k,m)L}$
\\
and on exit as
\\
{\bf source\_vector\_all(:,1:3,ispecp)}$|_{exit}= Y_{KL}=y_{(i,k,m)L}$
\\
Equation matrix contained in the sparse form in {\bf amat\_sp(1:nz)} which we denote as
\\
{\bf amat\_sp(1:nz)} $= A_{KK^\prime}$.
\\
Thus, the equation set which is solved is
\be{eqset}
\sum_{K^\prime} A_{KK^\prime} Y_{K^\prime L} = B_{KL}.
\ee
It is sufficient to consider equation matrix without collisions:
\\
{\bf
1659 !\\
1660 ! free flight: \\
1661 ! \\
1662       DO ipart=1,npassing+1 \\
1663         nz=nz+1 \\
1664 !        irow(nz)=k+ipart \\
1665 !        icol(nz)=k+ipart \\
1666 !        amat\_sp(nz)=delphim1 \\
1667         IF(colltest) nz\_ttmp=nz\_ttmp+1 \\
1668       ENDDO \\
1669 !\\
1670       DO ipart=1,npassing \\
1671         nz=nz+1 \\
1672 !        irow(nz)=k+ipart \\
1673 !        icol(nz)=k\_prev+ipart \\
1674 !        amat\_sp(nz)=-delphim1 \\
1675         IF(colltest) nz\_ttmp=nz\_ttmp+1 \\
1676       ENDDO \\
1677 !
}
\\
Here {\bf irow(nz)}$=K$, {\bf icol(nz)}$=K^\prime$, {\bf delphim1}$=(\varphi_i-\varphi_{i-1})^{-1}=\Delta\varphi^{-1}$,
and two types of matrix elements are filled:
$$
A_{KK}=\frac{1}{\Delta\varphi}, \qquad A_{KK_1}=-\frac{1}{\Delta\varphi},
$$
where $K=(i,k,m)$ and $K_1=(i-1,k,m)$
Thefore, Eq.~\eq{eqset} in this simplified case is
\be{sympleqset}
\frac{1}{\Delta\varphi}\left(y_{(i,k,m)L}-y_{(i-1,k,m)L}\right)=b_{(i,k,m)L}.
\ee
The r.h.s., $b_{(i,k,m)L}$, is given by the inout variable on entry:
\\
{\bf
4242             source\_vector(k+1:k+npassing+1,1)                                \& \\
4243                  =source\_vector(k+1:k+npassing+1,1)                          \& \\
4244                  +asource(m,1)/1.5d0*q\_rip(1:npassing+1,istep,1)             \& \\
4245                  *fact\_pos\_e(istep)\\
4246             source\_vector(k+1:k+npassing+1,2)                                \& \\
4247                  =source\_vector(k+1:k+npassing+1,2)                          \& \\
4248                  +asource(m,2)/1.5d0*q\_rip(1:npassing+1,istep,2)             \& \\
4249                  *fact\_pos\_e(istep)\\
4250             source\_vector(k+1:k+npassing+1,3)                                \& \\
4251                  =source\_vector(k+1:k+npassing+1,3)                          \& \\
4252                  +asource(m,3)/1.5d0*q\_rip(1:npassing+1,istep,1)             \& \\
4253                  *fact\_pos\_e(istep)
}
\\
This is a cubic order integration scheme (more operators from it were skipped). For our purpose it is enough
to simplify it to the lowest order changing {\bf 1.5d0} $-> 1$ and setting {\bf fact\_pos\_e(istep)}=1 (this is the
main case with exception of special $\varphi$ points):
\\
{\bf
source\_vector(k+1:k+npassing+1,1)=asource(m,1)*q\_rip(1:npassing+1,istep,1) \\
source\_vector(k+1:k+npassing+1,2)=asource(m,2)*q\_rip(1:npassing+1,istep,2) \\
source\_vector(k+1:k+npassing+1,3)=asource(m,3)*q\_rip(1:npassing+1,istep,1)
}
\\
Here {\bf k+1:k+npassing+1} is some range of $K$ values correponding to all $\eta$ levels with other indices fixed,
{\bf m}=$m$ is base function index, {\bf 1:npassing+1} is a range of $\eta$ levels $k$ and {\bf istep} is $\varphi$
step index $i$. Quantities
\\
{\bf asource(m,1)}$=a_m^{(1)}$, \\
{\bf asource(m,2)}$=a_m^{(3)}$, \\
{\bf asource(m,3)}$=a_m^{(2)}$ \\
are defined by Georg's equations (5.12)-(5.15) (note the remapping of indices 2 and 3).
Two other factors are defined by
\\
{\bf
4186       q\_rip(1:npassing+1,istep,1)=q\_rip\_1(1:npassing+1,istep)           \& \\
4187                                   *geodcu\_forw(istep)
}
\\
where lines {\bf 3246-3257} define {\bf geodcu\_forw(istep)}$=k_G |\nabla s|$ for $\varphi=\varphi_i$
\\
{\bf
1533     q\_rip\_1(1:npassing,istep)                                    \& \\
1534           =Vg\_vp\_over\_B(1:npassing,istep)-Vg\_vp\_over\_B(0:npassing-1,istep) \\
1535     q\_rip\_1(npassing+1,istep)=-Vg\_vp\_over\_B(npassing,istep) \\
1536     q\_rip(npassing+1,istep,2)=eta0-eta(npassing) \\
1537     q\_rip(1:npassing+1,istep,2)                                  \& \\
1538           =q\_rip(1:npassing+1,istep,2)                           \& \\
1539           *bhat\_mfl(istep)/h\_phi\_mfl(istep)
}
\\
Here, lines {\bf 1533-1535} together with \\
{\bf
1205     eta0=1.d0/bhat\_mfl(istep) \\
1206 ! \\
1207     DO i=0,npart \\
1208       subsq=1.d0-bhat\_mfl(istep)*eta(i) \\
1209       IF(subsq.GT.subsqmin) THEN \\
1210         npassing=i \\
1211         alambd(i,istep)=SQRT(subsq) \\
1212         Vg\_vp\_over\_B(i,istep)=alambd(i,istep)*eta0/h\_phi\_mfl(istep)            \& \\
1213 !                             *(4.d0*eta0-eta(i))*geodcu\_mfl(istep)/3.d0 \\
1214                              *(4.d0*eta0-eta(i))/3.d0 \\
1215       ELSE \\
1216         alambd(i,istep)=0.d0 \\
1217         Vg\_vp\_over\_B(i,istep)=0.d0 \\
1218       ENDIF \\
1219     ENDDO
}
\\
which define
$$
\mbox{{\bf Vg\_vp\_over\_B(k,i)}}= \frac{1}{h^\varphi}
\frac{|\lambda|}{3\hat B}\left(\frac{4}{\hat B}-\eta\right)
$$
for $\varphi=\varphi_i$ and $\eta=\eta_k$ result in
\\
$$
\mbox{{\bf q\_rip(k,i,1)}}=\frac{1}{h^\varphi}\int\limits_{\eta_{k-1}}^{\eta_k}\rd \eta q_1^\sigma,
$$
for $\varphi=\varphi_i$ where $q_1^\sigma$ is defined by Georg's (3.19) and
flux surface label is the normalized toroidal flux, $k_G \nabla\psi -> k_G \nabla s$ in Georg's (3.4).
\\
In turn, lines {\bf 1536-1539} together with \\
{\bf
1160   DO i=1,npart \\
1161     q\_rip(i,:,2)=eta(i)-eta(i-1) \\
1162   ENDDO
}
\\
define \\
$$
\mbox{{\bf q\_rip(k,i,2)}}=\frac{1}{h^\varphi}(\eta_k-\eta_{k-1})\hat B=\frac{1}{h^\varphi}
\int\limits_{\eta_{k-1}}^{\eta_k}\rd \eta q_3^\sigma
$$
for $\varphi=\varphi_i$ and $\sigma=1$ where $q_3^\sigma$ is defined by Georg's (3.20).
\\
Thus,
\be{bdef}
b_{(i,k,m)L}=\frac{a_m^{(L)}}{h^\varphi} \int\limits_{\eta_{k-1}}^{\eta_k}\rd \eta q_L^\sigma
\ee
for $\varphi=\varphi_i$, and $\sigma=1$.
\\
Integrating Georg's equation (3.26) with renotation $j=L$,
$$
\sigma \difp{f^{\sigma(L)}_m}{s} - \dots = a_m^{(L)} q_L^\sigma,
$$
over the $\eta$ band and presenting there
$$
\difp{}{s}=h^\varphi\difp{}{\varphi}
$$
we get
\be{georg3.26}
\difp{}{\varphi} \int\limits_{\eta_{k-1}}^{\eta_k}\rd \eta f^{\sigma(L)}_m - \dots =
\frac{a_m^{(L)}}{h^\varphi} \int\limits_{\eta_{k-1}}^{\eta_k}\rd \eta q_L^\sigma.
\ee
With the account of~\eq{bdef} we see that Eq.~\eq{sympleqset} is the lowest order discretization
over $\varphi$ of Eq.~\eq{georg3.26}, and, therefore
\be{soldef}
\mbox{\bf source\_vector\_all(:,1:3,ispecp)}|_{exit}=Y_{KL}
=y_{(i,k,m)L}= \int\limits_{\eta_{k-1}}^{\eta_k}\rd \eta \left .f^{\sigma(L)}_m\right|_{\varphi=\varphi_i}
\ee
with $k_G \nabla\psi -> k_G \nabla s$ and remapped drive indices.

\section{Flux matrix}
Axisymmetric flux matrix is computed after calling {\bf solve\_eqs} as follows
\\
{\bf
3301        qflux\_allspec=2.0d0*qflux\_allspec ! Caution!!! factor 2 is not needed!!! \\
3302        qflux\_symm\_allspec=qflux\_allspec
}
\\
and within {\bf solve\_eqs} quantity {\bf qflux\_allspec} is computed as
\\
{\bf
4146       qflux=0.5d0*REAL(MATMUL(CONJG(flux\_vector),source\_vector\_all(:,1:3,ispecp)),dp) \\
4147       qflux\_allspec(:,:,ispecp,ispec)=qflux
}
\\
Thus, for real matrices we can write directly in case of a single species ({\bf ispec=ispecp}=1)
\\
{\bf
qflux\_symm\_allspec = MATMUL(flux\_vector,source\_vector\_all(:,1:3,ispecp))
}
\\
We denote for a single species case, {\bf ispec=ispecp}=1,
\\
{\bf qflux\_symm\_allspec(1:3,1:3,ispecp,ispec)}$=q_{L,L^\prime}$
\\
and
\\
{\bf flux\_vector(1:3,:)}$=C_{LK}=c_{L(i,k,m)}$
\\
Thus with the account of~\eq{soldef} we get
\be{gendef_qflux}
q_{L,L^\prime} = \sum_K C_{LK} Y_{KL^\prime}=\sum_{i,k,m} c_{L(i,k,m)} y_{(i,k,m)L^\prime}.
\ee
Quantity $C_{LK}$ is determined by
\\
{\bf
4222         flux\_vector(1,k+1:k+npassing+1) =                                     \& \\
4223               step\_factor\_p*weightlag(1,m)*convol\_flux(1:npassing+1,istep) \\
4224         flux\_vector(1,k+npassing+2:k+2*npassing+2)=                           \& \\
4225               step\_factor\_m*weightlag(1,m)*convol\_flux(npassing+1:1:-1,istep) \\
4226 ! \\
4227         flux\_vector(2,k+1:k+npassing+1) =                                     \& \\
4228               step\_factor\_p*weightlag(2,m)*convol\_curr(1:npassing+1,istep) \\
4229         flux\_vector(2,k+npassing+2:k+2*npassing+2)=                           \& \\
4230              -step\_factor\_m*weightlag(2,m)*convol\_curr(npassing+1:1:-1,istep) \\
4231 ! \\
4232         flux\_vector(3,k+1:k+npassing+1) =                                     \& \\
4233               step\_factor\_p*weightlag(3,m)*convol\_flux(1:npassing+1,istep) \\
4234         flux\_vector(3,k+npassing+2:k+2*npassing+2) =                          \& \\
4235               step\_factor\_m*weightlag(3,m)*convol\_flux(npassing+1:1:-1,istep) \\
}
\\
Quantities {\bf step\_factor\_p} and {\bf step\_factor\_m} are computed in lines {\bf 4199 - 4215}
for high order integration scheme over $\varphi$. In the lowest order scheme they can be set simply to
the step size $\Delta \varphi$, see also
\\
{\bf
5154   delt\_pos(ibeg+1:iend)=phi\_mfl(ibeg+1:iend)-phi\_mfl(ibeg:iend-1) \\
5155   fact\_pos\_b=1.d0 \\
5156   fact\_pos\_e=1.d0
}
\\
and
\\
{\bf
5418   delt\_neg(ibeg:iend-1)=delt\_pos(ibeg+1:iend) \\
5419   fact\_neg\_b=fact\_pos\_e \\
5420   fact\_neg\_e=fact\_pos\_b \\
}
\\
Arrays {\bf weightlag} are defined as follows
$$
\mbox{\bf weightlag(1,m)}=\frac{1}{4} b^{(1)}_m
$$
$$
\mbox{\bf weightlag(2,m)}=\frac{1}{4} b^{(3)}_m
$$
$$
\mbox{\bf weightlag(3,m)}=\frac{1}{4} b^{(2)}_m
$$
where $b^{(L)}_m$ are defined by Georg's Eqs. (3.53) - (3.55).
\\
We introduce a general notation ${\cal C}_k^{(L)}(\varphi)$ such that
$$
{\cal C}_k^{(1)}(\varphi_i)={\cal C}_k^{(3)}(\varphi_i)=\mbox{\bf convol\_flux(k,i)}
$$
and
$$
{\cal C}_k^{(2)}(\varphi_i)=\mbox{\bf convol\_curr(k,i)}.
$$
With this we can present (ignoring remapping of drive indices $L$)
\be{CLK_def}
C_{LK}=c_{L(i,k,m)}=\frac{1}{4}\Delta\varphi b^{(L)}_m {\cal C}_k^{(L)}(\varphi_i).
\ee
Substituting~\eq{CLK_def} and~\eq{soldef} in~\eq{gendef_qflux} we get
\be{approxqflux}
q_{L,L^\prime} = \frac{1}{4}\sum_{i,k,m,\sigma}\Delta\varphi b^{(L)}_m {\cal C}_k^{(L)}(\varphi_i)
\int\limits_{\eta_{k-1}}^{\eta_k}\rd \eta \left .f^{\sigma(L^\prime)}_m\right|_{\varphi=\varphi_i}
\approx
\frac{1}{4} \oint \rd \varphi \sum_{k,m,\sigma} b^{(L)}_m {\cal C}_k^{(L)}(\varphi)
\int\limits_{\eta_{k-1}}^{\eta_k}\rd \eta f^{\sigma(L^\prime)}_m .
\ee
Here summation over $\sigma$ appears because it is implicitly included in $k$ summation in~\eq{gendef_qflux}.
\\
The easiest of ${\cal C}_k^{(L)}$ functions is
\\
{\bf
1541     convol\_curr(1:npassing+1,istep)=bhat\_mfl(istep)/h\_phi\_mfl(istep)
}
\\
It can be checked from the definition for the extended $k$ range (lines {\bf 4227 - 4230}) that
$$
{\cal C}_k^{(2)}(\varphi)=\frac{\sigma \hat B}{h^\varphi}=\frac{1}{h^\varphi} q_3^\sigma
$$
with $q_3^\sigma$ given by Georg's (3.20). Substituting this in~\eq{approxqflux} we get
\bea{currents}
q_{2,L^\prime} &\approx& \frac{1}{4} \oint \frac{\rd \varphi}{h^\varphi} \sum_{k,m,\sigma} b^{(3)}_m q_3^\sigma
\int\limits_{\eta_{k-1}}^{\eta_k}\rd \eta f^{\sigma(L^\prime)}_m
= \frac{1}{4} \oint \rd s \sum_{m,\sigma} b^{(3)}_m q_3^\sigma
\int\limits_{0}^{1/\hat B}\rd \eta f^{\sigma(L^\prime)}_m
\nonumber \\
&=& -\frac{1}{4} \oint \rd s \sum_{m,\sigma} b^{(3)}_m
\int\limits_{0}^{1/\hat B}\rd \eta f^{\sigma(L^\prime)}_m q_3^{-\sigma}.
\eea
Much more complicated is ${\cal C}_k^{(1)}$. It is given by
\\
{\bf
4188       convol\_flux(1:npassing+1,istep)=convol\_flux\_0(1:npassing+1,istep) \& \\
4189                                      *geodcu\_back(istep)
}
\\
In the axisymmetric case
\\
{\bf geodcu\_back = geodcu\_forw}$ = k_G |\nabla s|$
\\
The other factor is computed as
\\
{\bf
1494     coef\_cf=(1.d0,0.d0)/bhat\_mfl(istep)**2/h\_phi\_mfl(istep) \\
1495     convol\_flux\_0(1:npassing+1,istep)                                        \& \\
1496            =(convol\_polpow(0,1:npassing+1)+convol\_polpow(2,1:npassing+1))    \&\\
1497            *coef\_cf
}
\\
Here we need some intermediate results. Quantity {\bf vrecurr} computed on lines {\bf 1287 - 1322} is defined as
$$
\mbox{\bf vrecurr(n,m,k)}
= \frac{1}{m!}\int\limits_{\lambda_k}^{\lambda_{k-1}}\rd\lambda \lambda^n (\lambda-\lambda_k)^m,
$$
where $\lambda_k=\lambda(\eta_k)=\sqrt{1-\hat B \eta_k}$.
\\
Quantity {\bf convol\_polpow} computed on lines {\bf 1436-1442} and {\bf 1453-1465} is defined via {\bf vrecurr}
as
\\
{\bf
1440         convol\_polpow(0:legmax,i-1:i+2)=convol\_polpow(0:legmax,i-1:i+2)      \& \\
1441                  +MATMUL(vrecurr(0:legmax,0:3,i),bvec\_lapack)       !!!term[3]
}
\\
where {\bf bvec\_lapack} according to the comments on lines {\bf 1417-1426} is defined in such a way that
$$
\left.\difp{^m f}{\lambda^m}\right|_{\lambda=\lambda_k}=\sum\limits_{j=1}^4
\mbox{\bf bvec\_lapack(m+1,j)}\int\limits_{\eta_{k+j-3}}^{\eta_{k+j-2}}\rd \eta f.
$$
Respectively
\bea{bandint}
\int\limits_{\lambda_k}^{\lambda_{k-1}}\rd\lambda \lambda^n f
&\approx &
\int\limits_{\lambda_k}^{\lambda_{k-1}}\rd\lambda \lambda^n \sum\limits_{m=0}^3 \frac{(\lambda-\lambda_k)^m}{m!}
\left.\difp{^m f}{\lambda^m}\right|_{\lambda=\lambda_k}
\\
&=&
\sum\limits_{j=1}^4 \mbox{\bf MATMUL(vrecurr(n,0:3,k),bvec\_lapack(1:4,j))}
\int\limits_{\eta_{k+j-3}}^{\eta_{k+j-2}}\rd \eta f.
\nonumber
\eea
In lines {\bf 1440-1441} summation of integrals~\eq{bandint} over bands results in
$$
\int\limits_{-1}^{1}\rd\lambda \lambda^n f
=
\sum_{k,\sigma} \mbox{\rm convol\_polpow(n,k)}\; \sigma^n \int\limits_{\eta_{k}}^{\eta_{k-1}}\rd \eta f^\sigma.
$$
According to the lines {\bf 1494-1497} we have
\bea{convol_flux_0}
\sum_{k,\sigma} \mbox{\bf convol\_flux\_0(k,i)} \int\limits_{\eta_{k}}^{\eta_{k-1}}\rd \eta
\left. f^\sigma\right|_{\varphi=\varphi_i}
&=&
\frac{1}{\hat B^2 h^\varphi} \int\limits_{-1}^{1}\rd\lambda (1+\lambda^2) f
=
\frac{1}{h^\varphi}\sum_\sigma \int\limits_0^{1/\hat B}\rd \eta\frac{1+\lambda^2}{2|\lambda|\hat B} f^\sigma
\nonumber \\
&=&
-\frac{1}{h^\varphi} \sum_\sigma \int\limits_0^{1/\hat B}\rd \eta f^\sigma \difp{}{\eta}
\left(\frac{|\lambda|}{\hat B}\frac{\hat V_G}{|\nabla\psi|k_G}\right),
\nonumber
\eea
where $\hat V_G$ is given by Georg's (3.4).
Using now {\bf convol\_flux = convol\_flux\_0} $k_G |\nabla s|$ we rewrite~\eq{convol_flux_0} to
\bea{convol_flux_0_rewr}
\sum_{k,\sigma} {\cal C}_k^1 \int\limits_{\eta_{k}}^{\eta_{k-1}}\rd \eta \left. f^\sigma\right|_{\varphi=\varphi_i}
&=&
\sum_{k,\sigma} \mbox{\bf convol\_flux\_0(k,i)} \int\limits_{\eta_{k}}^{\eta_{k-1}}\rd \eta
\left. f^\sigma\right|_{\varphi=\varphi_i}
=
-\frac{1}{h^\varphi} \sum_\sigma \int\limits_0^{1/\hat B}\rd \eta f^\sigma q_1^{-\sigma},
\eea
where we used symmetry $q_1^{-\sigma}=q_1^{\sigma}$ and $q_1^\sigma$ has a re-defined geodesic curvature term
$k_G |\nabla \psi| -> k_G |\nabla s|$. Thus the general expression retains the strucure of~\eq{currents},
\bea{qflux_gen}
q_{L_\ast,L_\ast^\prime}
&=& -\frac{1}{4} \oint \rd s \sum_{m,\sigma} b^{(L)}_m
\int\limits_{0}^{1/\hat B}\rd \eta f^{\sigma(L^\prime)}_m q_L^{-\sigma}.
\eea
Here we introduced indices $L_\ast=L_\ast(L)$ and $L_\ast^\prime=L_\ast(L^\prime)$
in order to show index re-mapping explicitly,
\be{indremap}
L_\ast(1)=1, \qquad L_\ast(2)=3, \qquad L_\ast(3)=2.
\ee
Using the definition of flux surface average
$$
\langle A \rangle = \left(\oint \frac{\rd s}{\hat B}\right)^{-1} \oint \frac{\rd s}{\hat B} A,
$$
we can re-write~\eq{qflux_gen} as follows,
\bea{qflux_gen_flaver}
q_{L_\ast,L_\ast^\prime}
&=& - \sum_{m,\sigma} b^{(L)}_m
\left\langle \frac{\hat B}{4}\int\limits_{0}^{1/\hat B}\rd \eta f^{\sigma(L^\prime)}_m q_L^{-\sigma}\right\rangle
\oint \frac{\rd s}{\hat B}.
\eea

%%%%%%%%%%%%%%%%%%%%%%%%%%%%%%%%%%%%%%%%%%%%%%%%%%%%%%%%%%%%%%%%%%%%%%%%
%%%%%%%%%%%%%%%%%%%%%%%%%%%%%%%%%%%%%%%%%%%%%%%%%%%%%%%%%%%%%%%%%%%%%%%%
\section{Dimensional coefficients}
Comparing~\eq{qflux_gen_flaver} with Georg's Eq.(3.51) and using there $l_c/\tau_{ee}=v_{te}\rightarrow v_{T\alpha}$
(see definition of $l_c$ before Georg's Eq.(3.29) on page 21) we can write for a single species
\be{Lcoefviaqflux}
L_{LL^\prime}^\alpha=-n_\alpha v_{T\alpha} \beta_L \beta_{L^\prime}
\left(\oint \frac{\rd s}{\hat B}\right)^{-1}
q_{L_\ast,L_\ast^\prime}
\ee
with re-defined $\beta_L$ as compared to Georg's (3.18) because of different surface label ($\psi -> s$),
\be{betas}
\beta_1=\beta_2=\frac{\rho_\alpha}{\langle|\nabla s|\rangle}, \qquad \beta_3=1.
\ee
\red{
Here
\be{vtandrho}
\rho_\alpha = \frac{v_{T\alpha}}{\omega_{c\alpha}^{\rm ref}},
\qquad
v_{T\alpha} = \sqrt{\frac{2 T_\alpha}{m_\alpha}},
\qquad
\omega_{c\alpha}^{\rm ref} = \frac{e_\alpha B_{\rm ref}}{m_\alpha c},
\ee
$e_\alpha$, $m_\alpha$ and $B_{\rm ref}$ are species charge and mass and the reference magentic field, respectively.
}
We notice now that definitions of $A_3$ and $I_3$ in Winny's paper given there by Eqs. (4) and (11)
would have an opposite sign with respect to definitions of Georg, Eqs. (3.11) and (3.40),
if electron charge there retains its sign, $e<0$. However, negative sign is not possible there because
this would give the wrong sign of last terms in (3.9) and (3.10). Therefore $e$ must be a (positive) elementary
charge in Georg's Eqs. (3.11) and (3.40) what is in agreement with Winny's definitions. This points to the
inconsistency in Georg's notation because electron cyclotron frequency defined above (3.4) is positive.
Thus, correction of this inconsistency would mean just using the definitions of $A_3$ and $I_3$ Winny what
has no consequences for {\bf qflux}. Then, the only important difference
with Georg's quantities is that they are defined via normalized magnetic field while Winny's are not.
With this, diffusion coefficients $D_{jk}$ defined in Winny's paper by Eq. (14) are expressed via {\bf qflux} as
follows
\be{difcoefs}
D_{LL^\prime}^\alpha=-v_{T\alpha} \bar\beta_L \bar\beta_{L^\prime}
\left(\oint \frac{\rd s}{\hat B}\right)^{-1}
q_{L_\ast,L_\ast^\prime},
\ee
where $\bar \beta_{1,2}=\beta_{1,2}$ and $\bar \beta_3= \beta_3 B_{\rm ref} = B_{\rm ref}$.
Coefficients $D_{LL^\prime}$ for $L,L^\prime=1,2$ normalized by the plateau value
\be{dplateau}
D_p^\alpha = \frac{\pi v_{T\alpha} \rho_\alpha^2}{16\iota R_0}
\ee
are
\be{platnorm}
\frac{D_{LL^\prime}^\alpha}{D_p^\alpha}=-\frac{16\iota R}{\pi\langle|\nabla s|\rangle^2}
\left(\oint \frac{\rd s}{\hat B}\right)^{-1}
q_{L_\ast,L_\ast^\prime}, \qquad L,L^\prime=1,2.
\ee
\red{
Value of $R_0$ in Eq~\eq{dplateau} corresponds to the major radius of the magnetic axis,
and $\iota=\iota(s)$ is a local value.
}
Cross coefficients with the third drive or flux normalized by the ion-like Ware pitch coefficient
\red{
\be{wareion}
D_{13}^{AX,\alpha}=\frac{c T_\alpha B_\varphi}{e_\alpha \sqrt{g}B^\vartheta}
=\frac{v_{T\alpha}\rho_\alpha B_\varphi B_{\rm ref}}{2\sqrt{g}B^\vartheta}
=\frac{v_{T\alpha}\rho_\alpha B_\varphi B_{\rm ref}}{2\iota \psi_a \langle|\nabla s|\rangle},
\ee
as follows
\be{warenorm}
\frac{D_{LL^\prime}^\alpha}{D_{13}^{AX,\alpha}}=-\frac{2\iota \psi_a}{B_\varphi}
\left(\oint \frac{\rd s}{\hat B}\right)^{-1}
q_{L_\ast,L_\ast^\prime}, \qquad L\ne 3, L^\prime=3 \quad\mbox{or}\quad L=3; L^\prime \ne 3.
\ee
}

\noindent
For normalization of $D_{33}^\alpha$ coefficients we use Lorentz conductivity in the homegeneous magnetic field
(conductivity taking only electron-ion collisions for infinite mass ions with charge number $Z_{\rm eff}$)
which is given by Georg's Eq.(6.28),
\be{sigmaL}
\sigma^L_{\rm NEO}=\frac{32}{3\pi}\frac{n_e e^2}{m_e}\frac{\tau_ee}{Z_{\rm eff}}=
\frac{32}{3\pi}\frac{n_e e^2}{m_e}\frac{l_c}{v_{Te}Z_{\rm eff}},
\ee
where $l_c$ is defined by Eq.(16) of Winny and in the text below Eq.(16). Here one should take it for electrons.
In the homogeneous magnetic field, $B=B_{\rm ref}$, using Winny's definitions of forces and fluxes one gets
$$
j_\parallel = -\frac{n_e e^4 D_{33}^e E_\parallel}{T_e B^2_{\rm ref}}=\sigma E_\parallel.
$$
Substituting here $\sigma = \sigma^L_{\rm NEO}$ from~\eq{sigmaL} one gets
\be{D33L}
D_{33}^{L}=-\frac{16}{3\pi} B^2_{\rm ref} v_{Te} \frac{l_c}{Z_{\rm eff}}.
\ee
Thus the normalized parallel flow coefficient is
\be{D33norm}
\frac{D^\alpha_{33}}{Z_{\rm eff} D_{33}^{L}}=\frac{3\pi}{16\, l_c} \sqrt{\frac{m_e T_\alpha}{m_\alpha T_e}}
\left(\oint \frac{\rd s}{\hat B}\right)^{-1}q_{L_\ast(3),L_\ast(3)},
\qquad q_{L_\ast(3),L_\ast(3)}=q_{2,2}.
\ee
For electrons this quantity corresponds to $\gamma_E / Z_{\rm eff}$ where $\gamma_E$ is given by Eq.(6.47) of Georg
and its values are given in Table 6.1 there (this is actually the quantity tabulated in the original paper of
Spitzer and H\"arm).

%%%%%%%%%%%%%%%%%%%%%%%%%%%%%%%%%%%%%%%%%%%%%%%%%%%%%%%%%%%%%%%%%%%%%%%%
%%%%%%%%%%%%%%%%%%%%%%%%%%%%%%%%%%%%%%%%%%%%%%%%%%%%%%%%%%%%%%%%%%%%%%%%
\section{Transport coefficients computed by the multi-species version of NEO-2}

In a multi-species plasma the thermodynamic fluxes $I_k^\alpha$,
\bea{multispec:thermodyn_fluxes}
I_1^\alpha &=& \Gamma_\alpha = \left\langle \int {\rd^3}v v_{g,\alpha}^r f_{\alpha 1} \right\rangle, \non
I_2^\alpha &=& \frac{Q_\alpha}{T_\alpha} = \frac{1}{T_\alpha}\left\langle \int {\rd^3}v \frac{m_\alpha v^2}{2} v_{g,\alpha}^r f_{\alpha 1} \right\rangle, \non
I_3^\alpha &=& n_\alpha \left\langle V_{\parallel,\alpha} B \right\rangle = \left\langle B \int {\rd^3}v v_\parallel f_{\alpha 1} \right\rangle,
\eea
are linked to the thermodynamic forces $A_k^{\alpha'}$,
\be{multispec:thermodyn_forces}
A_1^{\alpha'} = \frac{1}{n_{\alpha'}} \difp{n_{\alpha'}}{r} - \frac{e_{\alpha'} E_r}{T_{\alpha'}} -\frac{3}{2 T_{\alpha'}} \difp{T_{\alpha'}}{r}, \qquad
A_2^{\alpha'} = \frac{1}{T_{\alpha'}} \difp{T_{\alpha'}}{r}, \qquad
A_3^{\alpha'} = \frac{e_{\alpha'} \left\langle E_\parallel B \right\rangle}{T_{\alpha'} \left\langle B^2 \right\rangle},
\ee
via the relation,
\be{multispec:thermodyn_fluxes_forces}
I_j^\alpha = -n_\alpha \sum\limits_{\alpha'} \sum\limits_{k=1}^{3} D_{jk}^{\alpha \alpha'} A_{k}^{\alpha'}.
\ee
Here, $v_{g,\alpha}^r$ and $f_{\alpha 1}$ are the radial component of the guiding center drift velocity and
the solution of the multi-species drift kinetic equation (DKE) system for species $\alpha$, respectively.
The species particle flux density, heat flux density, parallel flow velocity, density, temperature, mass and charge
are denoted by $\Gamma_\alpha$, $Q_\alpha$, $V_{\parallel,\alpha}$, $n_\alpha$, $T_\alpha$, $m_\alpha$ and $e_\alpha$, respectively,
whereas $E_r$ is the radial electric field and $E_\parallel$ is the parallel electric field.
In NEO-2 the multi-species transport coefficients $D_{jk}^{\alpha\alpha'}$ are calculated from the solution of the
multi-species DKE system,
\bea{multispec:DKEsystem}
\hat{L}_{\rm D}^{\alpha_0} f_{(k \alpha')}^{\alpha_0} + \sum\limits_{\alpha''} \hat{L}_{\rm C,I}^{\alpha_0 \alpha''} f_{(k \alpha')}^{\alpha''} &=& 0 \non
\vdots \hphantom{\hat{L}_{\rm D}^{\alpha} f_{(k \alpha')}^{\alpha} + \sum\limits_{\alpha''} \hat{L}_{\rm C,I}^{\alpha\alpha''}} &\vdots& \vdots \non
\hat{L}_{\rm D}^{\alpha} f_{(k \alpha')}^{\alpha} + \sum\limits_{\alpha''} \hat{L}_{\rm C,I}^{\alpha\alpha''} f_{(k \alpha')}^{\alpha''} &=& 0 \non
\vdots \hphantom{\hat{L}_{\rm D}^{\alpha} f_{(k \alpha')}^{\alpha} + \sum\limits_{\alpha''} \hat{L}_{\rm C,I}^{\alpha\alpha''}} &\vdots& \vdots \non
\hat{L}_{\rm D}^{\alpha_{m-1}} f_{(k \alpha')}^{\alpha_{m-1}} + \sum\limits_{\alpha''} \hat{L}_{\rm C,I}^{\alpha_{m-1} \alpha''} f_{(k \alpha')}^{\alpha''} &=& 0 \non
\hat{L}_{\rm D}^{\alpha'} f_{(k \alpha')}^{\alpha'} + \sum\limits_{\alpha''} \hat{L}_{\rm C,I}^{\alpha'\alpha''} f_{(k \alpha')}^{\alpha''} &=& \tilde{q}^{\sigma}_{(k \alpha')} f_{{\rm M},\alpha'} \non
\hat{L}_{\rm D}^{\alpha_{m+1}} f_{(k \alpha')}^{\alpha_{m+1}} + \sum\limits_{\alpha''} \hat{L}_{\rm C,I}^{\alpha_{m+1} \alpha''} f_{(k \alpha')}^{\alpha''} &=& 0 \non
\vdots \hphantom{\hat{L}_{\rm D}^{\alpha} f_{(k \alpha')}^{\alpha} + \sum\limits_{\alpha''} \hat{L}_{\rm C,I}^{\alpha\alpha''}} &\vdots& \vdots \non
\hat{L}_{\rm D}^{\alpha_n} f_{(k \alpha')}^{\alpha_n} + \sum\limits_{\alpha''} \hat{L}_{\rm C,I}^{\alpha_n \alpha''} f_{(k \alpha')}^{\alpha''} &=& 0 ,
\eea
where the solution $f_{\alpha 1}$ has beeen expressed in terms of a superposition of species thermodynamic forces,
\be{multispec:fa1expand}
f_{\alpha 1} = \sum\limits_{\alpha'} \sum\limits_{k=1}^{3} f_{(k \alpha')}^{\alpha} A_k^{\alpha'},
\ee
because the problem is linear in the driving forces.
Here, $\hat{L}_{\rm D}^{\alpha}$ and $\hat{L}_{\rm C,I}^{\alpha\alpha''}$ are the differential part of the multi-sepcies DKE system and
the integral part of the Coulomb collision operator, respectively, and the quantity $\tilde{q}^{\sigma}_{(k \alpha)}$ can be expressed
in the compact notation,
\be{multispec:qktilde}
\tilde{q}^{\sigma}_{(k \alpha)} = v_{T \alpha} \bar{\beta}_k^\alpha x^{2k-5\delta_{3k}} |\lambda| q^{\sigma}_{k},
\ee
where $v_{T \alpha}=\sqrt{2 T_\alpha / m_\alpha}$ is the species thermal velocity, $x=v/v_{T \alpha}$,
$\bar{\beta}_1^\alpha = \bar{\beta}_2^\alpha = \rho_{\alpha 0} / \left\langle |\nabla s| \right\rangle$ and
$\bar{\beta}_3^\alpha = B_{\rm ref}$.
The species Larmor radius and normalized toroidal flux are $\rho_{\alpha 0}$ and $s$, respectively,
and the quantities $q^{\sigma}_{k}$ are given by Eq.~(3.19) and Eq.~(3.20) of Georg's thesis
using a re-defined geodesic curvature term $k_G |\nabla \psi| -> k_G |\nabla s|$.
Analogous to Eq.~(3.43) of Georg's thesis the multi-species transport coefficients can be expressed as
\be{multispec:difcoefs}
D_{jk}^{\alpha \alpha'} = \frac{1}{n_\alpha} \left\langle \int {\rd}^3v
\tilde{q}^{-\sigma}_{(j \alpha)} f_{(k \alpha')}^{\alpha} \right\rangle.
\ee
In the code the solution functions $f_{(k \alpha')}^{\alpha}$ are normalized by the
species coefficient $\bar{\beta}_k^{\alpha'}$ appearing in the driving forces
$\tilde{q}^{\sigma}_{(k \alpha')}$ in the right-hand side of the multi-species DKE system,
\be{multispec:fa1norm}
\hat{f}_{(k \alpha')}^{\alpha} = f_{(k \alpha')}^{\alpha} / \bar{\beta}_k^{\alpha'},
\ee
in order to make the multi-species DKE system independent of this coefficient.
Furthermore, the normalized solution function is expressed in terms of test functions
$\phi_m$ in order to resolve the energy dependence of the solution,
\bea{multispec:fa1basisfun}
\hat{f}_{(k \alpha')}^{\alpha}(r,s,x,\lambda)
&=&
f_{\alpha 0}(r,x) \sum\limits_{m} f_{m,(k \alpha')}^{\alpha}(r,s,\lambda) \phi_m(x) \non
&=& \frac{n_\alpha}{\pi^{3/2} v_{T \alpha}^3} e^{-x^2}
\sum\limits_{m} f_{m,(k \alpha')}^{\alpha}(r,s,\lambda) \phi_m(x).
\eea
Using Eq.~\eqref{multispec:qktilde}, Eq.~\eqref{multispec:fa1norm}, Eq.~\eqref{multispec:fa1basisfun}
and the expression for the volume element in velocity space,
\be{multispec:volumeelement}
\int {\rd}^3v = \pi v_{T\alpha}^3 \hat{B} \sum\limits_{\sigma=\pm 1}
\int\limits_0^\infty {\rd}x x^2 \int\limits_0^{1/\hat{B}} {\rd}\eta \frac{1}{|\lambda|},
\ee
the multi-species transport coefficients~\eqref{multispec:difcoefs} can be re-written as,
\be{multispec:difcoefs_full}
D_{jk}^{\alpha \alpha'} = v_{T \alpha} \bar{\beta}_j^{\alpha} \bar{\beta}_k^{\alpha'}
\sum\limits_m \sum\limits_{\sigma=\pm 1}
\underbrace{\left( \frac{4}{\sqrt{\pi}} \int\limits_0^\infty x^{2j+2-5\delta_{3j}} e^{-x^2} \phi_m(x) \right)}_{\equiv b_m^{(j)}}
\left\langle \frac{\hat{B}}{4} \int\limits_0^{1/\hat{B}} {\rd}\eta q_j^{-\sigma} f_{m,(k \alpha')}^{\alpha} \right\rangle.
\ee
Analogous to Eq.~\eqref{difcoefs}, the multi-species diffusion coefficients $D_{jk}^{\alpha \alpha'}$
can be expressed in terms of {\bf qflux} as follows,
\be{multispec:difcoefs_final}
D_{LL^\prime}^{\alpha \alpha'} =
-v_{T\alpha} \bar\beta_{L}^{\alpha} \bar\beta_{L^\prime}^{\alpha'}
\left(\oint \frac{\rd s}{\hat B}\right)^{-1}
q_{L_\ast,L_\ast^\prime}^{\alpha \alpha'}.
\ee
In the NEO-2 subroutine \verb|compute_Dijab| of \verb|ntv_mod.f90|
dimensional and normalized transport coefficients are computed.
Normalizations of transport coefficients are introuced in the next subsection.
Species particle flux density $\Gamma_\alpha$, species heat flux denstity $Q_\alpha$ and
species parallel flow $\left\langle V_{\parallel,\alpha} B \right\rangle$ are
conmputed by the NEO-2 subroutines \verb|compute_Gamma|, \verb|compute_Qflux| and
\verb|compute_ParFlow| of \verb|ntv_mod.f90| , respectively.

%%%%%%%%%%%%%%%%%%%%%%%%%%%%%%%%%%%%%%%%%%%%%%%%%%%%%%%%%%%%%%%%%%%%%%%%
\subsection{Normalized transport coefficients}

Coefficients $D_{LL^\prime}^{\alpha \alpha'}$ for $L,L^\prime=1,2$ normalized by the plateau value
$D_p^{\alpha_1}$ Eq.~\eq{dplateau} of the first species, $\alpha_1=1$ (usually electrons)
are
\be{multispec:platnorm}
\frac{D_{LL^\prime}^{\alpha \alpha'}}{D_p^{\alpha_1}}=
-\frac{16\iota R}{\pi\langle|\nabla s|\rangle^2}
\left(\oint \frac{\rd s}{\hat B}\right)^{-1}
\left(\frac{m_\alpha m_{\alpha'}}{m_{\alpha_1}^2}\right)
\left(\frac{v_{T \alpha}^2 v_{T \alpha'}}{v_{T \alpha_1}^3}\right)
\left(\frac{Z_{\alpha_1}^2} {Z_\alpha Z_{\alpha'}}\right)
q_{L_\ast,L_\ast^\prime}^{\alpha \alpha'}, \qquad L,L^\prime=1,2.
\ee
The cross coefficients for the third drive normalized by the ion-like Ware pitch coefficient
$D_{13}^{{\rm AX},\alpha_1}$ Eq.~\eq{wareion} of the first species, $\alpha_1=1$ (usually electrons),
are
\be{multispec:warenorm}
\frac{D_{LL^\prime}^{\alpha \alpha'}}{D_{13}^{{\rm AX},\alpha_1}}=
-\frac{2\iota \psi_a}{B_\varphi}
\left(\oint \frac{\rd s}{\hat B}\right)^{-1}
\left(\frac{m_\alpha}{m_{\alpha_1}}\right)
\left(\frac{v_{T \alpha}^2}{v_{T \alpha_1}^2}\right)
\left(\frac{Z_{\alpha_1}} {Z_\alpha}\right)
q_{L_\ast,L_\ast^\prime}^{\alpha \alpha'}, \qquad L\ne 3, L^\prime=3.
\ee
The cross coefficients for the third flux normalized by the ion-like Ware pitch coefficient
of the first species (usually electrons) are
\be{multispec:bootstrapnorm}
\frac{D_{LL^\prime}^{\alpha \alpha'}}{D_{13}^{{\rm AX},\alpha_1}}=
-\frac{2\iota \psi_a}{B_\varphi}
\left(\oint \frac{\rd s}{\hat B}\right)^{-1}
\left(\frac{m_{\alpha'}}{m_{\alpha_1}}\right)
\left(\frac{v_{T \alpha} v_{T \alpha'}}{v_{T \alpha_1}^2}\right)
\left(\frac{Z_{\alpha_1}} {Z_{\alpha'}}\right)
q_{L_\ast,L_\ast^\prime}^{\alpha \alpha'}, \qquad L= 3, L^\prime \ne 3.
\ee
The normalized parallel flow coefficient is
\be{multispec:D33norm}
\frac{D^{\alpha \alpha'}_{33}}{Z_{\rm eff} D_{33}^{L}}=\frac{3\pi}{16\, l_{c,\alpha_1}}
\sqrt{\frac{m_{\alpha_1} T_\alpha}{m_\alpha T_{\alpha_1}}}
\left(\oint \frac{\rd s}{\hat B}\right)^{-1}q_{L_\ast(3),L_\ast(3)}^{\alpha \alpha'},
\qquad q_{L_\ast(3),L_\ast(3)}^{\alpha \alpha'}=q_{2,2}^{\alpha \alpha'}.
\ee

%%%%%%%%%%%%%%%%%%%%%%%%%%%%%%%%%%%%%%%%%%%%%%%%%%%%%%%%%%%%%%%%%%%%%%%%
\subsection{Computation of the radial electric field}
In the NEO-2 subroutine \verb|compute_Er| of \verb|ntv_mod.f90| the radial electric field is evaluated
without or with account of the parallel electric field.
The radial electric field is evaluated from the toroidal rotation velocity and plasma
parameter profiles using the ideal MHD equilibrium equation for the ion component
(see Eq.~(4) of [Kasilov et al., Physics of Plasmas 21, 092506 (2014)]),
\be{multispec:idealMHDEq}
\nabla p_\alpha = e_\alpha n_\alpha \left( \bE + \frac{1}{c} \bV_\alpha \times \bB \right).
\ee
Taking the cross-product of Eq.~\eqref{multispec:idealMHDEq} with $\bB$ gives
the perpendicular component of the species flow velocity $\bV_{\alpha}$,
\bea{multispec:VperpSpec}
\bV_{\perp,\alpha}
&=&
\frac{c}{B} \bh \times \left( \frac{\nabla p_\alpha}{e_\alpha n_\alpha} - \bE \right) \non
&=&
\frac{c}{B} \left( \frac{1}{e_\alpha n_\alpha} \difp{p_\alpha}{r} - E_r \right) \bh \times {\nabla}r,
\eea
using $p_\alpha = p_\alpha(r)$ and $\bE = - \nabla\Phi(r)$.
In Boozer coordinates $(r,\tht_B,\ph_B)$ the contra-variant components of $\bV_{\perp,\alpha}$ are
\bea{multispec:VperpSpec_ctrvr}
V_{\perp,\alpha}^r &=& \bV_{\perp,\alpha} \cdot {\nabla}r = 0, \non
V_{\perp,\alpha}^{\tht_B}(r)&=& \bV_{\perp,\alpha} \cdot {\nabla}\tht_B =
\frac{c B_{\ph_B}}{\sqrt{g_B} B^2} \left(\frac{1}{e_\alpha n_\alpha} \difp{p_\alpha}{r}-E_r\right), \non
V_{\perp,\alpha}^{\ph_B}(r) &=& \bV_{\perp,\alpha} \cdot {\nabla}\ph_B =
-\frac{c B_{\tht_B}}{\sqrt{g_B} B^2} \left(\frac{1}{e_\alpha n_\alpha} \difp{p_\alpha}{r}-E_r\right).
\eea
Eq.~\eqref{multispec:VperpSpec} also implies that the radial contra-variant component of
the species flow velocity is zero,
\be{multispec:VrCtrvr}
V_\alpha^r = \bV_\alpha \cdot {\nabla}r =
\left( \bV_{\perp,\alpha} + \bh V_{\parallel,\alpha} \right) \cdot {\nabla}r = 0.
\ee
Furthermore, it is assumed that the species flow velocity is divergence-free,
\be{multispec:VDivFree}
\nabla \cdot \bV = 0.
\ee
The parallel component of the species flow velocity $V_{\parallel,\alpha}$ can be determined from
Eq.~\eqref{multispec:VDivFree} taking into account Eq.~\eqref{multispec:VperpSpec},
Eq.~\eqref{multispec:VrCtrvr} and the toroidal symmetry of the device,
\bea{multispec:VDivFree_1D}
\nabla \cdot \bV
&=&
\frac{1}{\sqrt{g_B}} \difp{}{r} \sqrt{g_B} \underbrace{V_\alpha^r}_{= \ 0} +
\frac{1}{\sqrt{g_B}} \difp{}{\tht_B} \sqrt{g_B} V_\alpha^{\tht_B} +
\frac{1}{\sqrt{g_B}} \underbrace{\difp{}{\ph_B} \sqrt{g_B} V_\alpha^{\ph_B}}_{\rightarrow \ 0} \non
&=&
\frac{1}{\sqrt{g_B}} \difp{}{\tht_B} \sqrt{g_B} \left(
\frac{c B_{\ph_B}}{\sqrt{g_B} B^2} \left( \frac{1}{e_\alpha n_\alpha} \difp{p_\alpha}{r} - E_r \right) +
\frac{B^{\tht_B}}{B} V_{\parallel,\alpha}
\right) \non
&=&
\frac{1}{\sqrt{g_B}} \difp{}{\tht_B} \underbrace{\sqrt{g_B} B^{\tht_B}}_{= \ f(r)} \left(
\frac{c B_{\ph_B}}{\sqrt{g_B} B^{\tht_B} B^2} \left( \frac{1}{e_\alpha n_\alpha} \difp{p_\alpha}{r} - E_r \right) +
\frac{V_{\parallel,\alpha}}{B} \right) \non
&\overset{!}{=} &  0,
\eea
as
\bea{multispec:VparSpec}
V_{\parallel,\alpha}
&=&
- \frac{c B_{\ph_B}}{\sqrt{g_B} B^{\tht_B} B}
\left( \frac{1}{e_\alpha n_\alpha} \difp{p_\alpha}{r} - E_r \right)
+ K(r) B \non
&=&
- \frac{c T_\alpha B_{\ph_B}}{e_\alpha \sqrt{g_B} B^{\tht_B} B}
\left( \frac{1}{p_\alpha} \difp{p_\alpha}{r} - \frac{e_\alpha E_r}{T_\alpha} \right)
+ K(r) B.
\eea
Using Eqs.~\eqref{multispec:VperpSpec_ctrvr} and~\eqref{multispec:VparSpec},
the radial electric field can be evaulated for a known (experimentally measuered)
flux surface averaged toroidal rotation velocity $\left\langle V_\alpha^\ph \right\rangle$
(activated for \verb|isw_Vphi_loc=0|).
Here $\ph$ denotes the geometric angle.
Firstly, the flux surface averaged toroidal rotation velocity in Boozer coordinates
$\left\langle V_\alpha^{\ph_B} \right\rangle$ is expressed  in terms of the
species parallel flow $\left\langle V_{\parallel,\alpha} B \right\rangle$,
\be{multispec:calc_Er_step1}
\left\langle V_\alpha^{\ph_B} \right\rangle =
-\frac{c B_{\tht_B}}{\sqrt{g_B} B^2} \left(\frac{1}{e_\alpha n_\alpha} \difp{p_\alpha}{r}-E_r\right) +
\frac{B^{\ph_B}}{B^2} \left\langle V_{\parallel,\alpha} B \right\rangle.
\ee
The species parallel flow can be evaluated in terms of transport coefficients
that are determined by the NEO-2 solution for the axisymmetric configuration,
\bea{multispec:calc_Er_step1a}
\left\langle V_{\parallel,\alpha} B \right\rangle
&=&
- \sum\limits_{\alpha'} \left(
D_{31}^{\alpha \alpha'} \left(-\frac{e_{\alpha'} E_r}{T_{\alpha'}}\right) +
D_{31}^{\alpha \alpha'} \frac{1}{p_{\alpha'}}\difp{p_{\alpha'}}{r} +{} \right. \non
& & {}+ \left.
\left( D_{32}^{\alpha \alpha'} - \frac{5}{2} D_{31}^{\alpha \alpha'}\right)
\frac{1}{T_{\alpha'}}\difp{T_{\alpha'}}{r} +
D_{33}^{\alpha \alpha'} \frac{e_{\alpha'}}{T_{\alpha'}}
\frac{\left\langle E_{\parallel} B  \right\rangle }{\left\langle B^2 \right\rangle}
\right) \non
&\overset{!}{=}&
\frac{B^2}{B^{\ph_B}} \left\langle V_\alpha^{\ph_B} \right\rangle +
\frac{c B_{\tht_B}}{\sqrt{g_B} B^{\ph_B}}
\left(\frac{1}{e_\alpha n_\alpha} \difp{p_\alpha}{r}-E_r\right),
\eea
which in turn yields following expression for the radial electric field,
\bea{multispec:calc_Er_step1b}
E_r
&=&
\left[
\frac{c B_{\tht_B}}{\sqrt{g_B} B^{\ph_B}} +
\sum\limits_{\alpha'} D_{31}^{\alpha \alpha'} \frac{e_{\alpha'}}{T_{\alpha'}}
\right]^{-1}
\left(
\frac{B^2}{B^{\ph_B}} \left\langle V_\alpha^{\ph_B} \right\rangle +
\frac{c T_\alpha B_{\tht_B}}{e_\alpha \sqrt{g_B} B^{\ph_B}}
\frac{1}{p_\alpha} \difp{p_\alpha}{r} \right. +{} \non
& & +{} \left.
\sum\limits_{\alpha'} \left[
D_{31}^{\alpha \alpha'} \frac{1}{p_{\alpha'}}\difp{p_{\alpha'}}{r} +
\left( D_{32}^{\alpha \alpha'} - \frac{5}{2} D_{31}^{\alpha \alpha'}\right)
\frac{1}{T_{\alpha'}}\difp{T_{\alpha'}}{r} +
D_{33}^{\alpha \alpha'} \frac{e_{\alpha'}}{T_{\alpha'}}
\frac{\left\langle E_{\parallel} B  \right\rangle }{\left\langle B^2 \right\rangle}
\right]
\right).
\eea
Secondly, we show that
$\left\langle V_\alpha^{\ph_B} \right\rangle = \left\langle V_\alpha^{\ph} \right\rangle$.
In a Boozer file the transformation function $G$ between Boozer coordinates and
symmetry flux coordinates (with respect to the geometric angle $\ph$ as a symmetry coordinate)
is generally given as
\bea{multispec:def_Gfun}
\ph - \ph_B &=& +\frac{2\pi}{N_p} l \overset{!}{=} 2\pi \difp{\psi_{\rm tor}}{s} G(s,\tht_B,\ph_B)
\quad {\rm for} \quad {\rm lab\_swi=9},\non
\ph - \ph_B &=& -\frac{2\pi}{N_p} l \overset{!}{=} 2\pi \difp{\psi_{\rm tor}}{s} G(s,\tht_B,\ph_B)
\quad {\rm for} \quad {\rm lab\_swi=else},
\eea
where $s=\psi_{\rm tor}/\psi_{\rm tor}^{\rm a}$ is the normalized toroidal flux
(typically used as a flux surface label in a Boozer file instead of the effective radius $r$),
$N_p$ is the number of field periods and the quantity $l$ is given in terms of Fourier coefficients,
\bea{multispec:def_lmn}
l &=& \sum\limits_{m,n}
\left(
l_{m,n}^{\rm c} \cos{\left(m\tht_B + n\ph_B\right)} +
l_{m,n}^{\rm s} \sin{\left(m\tht_B + n\ph_B\right)}
\right)
\quad {\rm for} \quad {\rm lab\_swi=9},\non
l &=&
\sum\limits_{m,n}
l_{m,n} \sin{\left(m\tht_B - n\ph_B\right)}
\quad {\rm for} \quad {\rm lab\_swi=else}.
\eea
Using Eq.~\eqref{multispec:def_Gfun} $\left\langle V_\alpha^{\ph} \right\rangle$
can be related to $\left\langle V_\alpha^{\ph_B} \right\rangle$,
\be{multispec:calc_Er_step2a}
\left\langle V_\alpha^{\ph} \right\rangle =
\left\langle V_\alpha^{\ph_B} \right\rangle +
2\pi \difp{\psi_{\rm tor}}{s}
\left\langle \difp{G(s,\tht_B)}{\tht_B} V_\alpha^{\tht_B} \right\rangle,
\ee
where (to the lowest order) $G$ is assumed to be independent of $\ph_B$ due to the smallness
of the perturbation field amplitude (quasilinear approximation).
The second term on the right-hand side of Eq.~\eqref{multispec:calc_Er_step2a} is zero
since $G$ is a periodic function of angles,
\bea{multispec:calc_Er_step2b}
\left\langle \difp{G(s,\tht_B)}{\tht_B} V_\alpha^{\tht_B} \right\rangle
&=&
\left(\int\limits_{0}^{2\pi} \rd\ph_B \int\limits_{0}^{2\pi} \rd\tht_B \sqrt{g_B}\right)^{-1}
\int\limits_{0}^{2\pi} \rd\ph_B \int\limits_{0}^{2\pi} \rd\tht_B
\underbrace{\sqrt{g_B} B^{\tht_B} K(r)}_{= f(r)} \difp{G}{\tht_B} \non
&=&
\left(\int\limits_{0}^{2\pi} \rd\ph_B \int\limits_{0}^{2\pi} \rd\tht_B \sqrt{g_B}\right)^{-1}
\sqrt{g_B} B^{\tht_B} K(r) \int\limits_{0}^{2\pi} \rd\ph_B
\underbrace{\int\limits_{0}^{2\pi} \rd\tht_B \difp{G}{\tht_B}}_{= 0} = 0,
\eea
and thus $\left\langle V_\alpha^{\ph} \right\rangle = \left\langle V_\alpha^{\ph_B} \right\rangle$.
The expression for the radial electric field finally becomes
\bea{multispec:Er_from_Vphiav}
E_r
&=&
\left[
\frac{c B_{\tht_B}}{\sqrt{g_B} B^{\ph_B}} +
\sum\limits_{\alpha'} D_{31}^{\alpha \alpha'} \frac{e_{\alpha'}}{T_{\alpha'}}
\right]^{-1}
\left(
\frac{B^2}{B^{\ph_B}} \left\langle V_\alpha^{\ph} \right\rangle +
\frac{c T_\alpha B_{\tht_B}}{e_\alpha \sqrt{g_B} B^{\ph_B}}
\frac{1}{p_\alpha} \difp{p_\alpha}{r} \right. +{} \non
& & +{} \left.
\sum\limits_{\alpha'} \left[
D_{31}^{\alpha \alpha'} \frac{1}{p_{\alpha'}}\difp{p_{\alpha'}}{r} +
\left( D_{32}^{\alpha \alpha'} - \frac{5}{2} D_{31}^{\alpha \alpha'}\right)
\frac{1}{T_{\alpha'}}\difp{T_{\alpha'}}{r} +
D_{33}^{\alpha \alpha'} \frac{e_{\alpha'}}{T_{\alpha'}}
\frac{\left\langle E_{\parallel} B  \right\rangle }{\left\langle B^2 \right\rangle}
\right]
\right).
\eea
For a simple plasma with a single ion species ($\alpha=i$) an analytical solution for
the bootstrap transport coefficient $D_{31}^{ii}$ is known
(see Eq.~(10) of [Kasilov et al., Physics of Plasmas 21, 092506 (2014)]),
\be{multispec:D31_analyt}
D_{31}^{ii} = \frac{c T_i B_{\ph_B}}{e_i \sqrt{g_B} B^{\tht_B}}.
\ee
If the contribution from the parallel electric field is negligible small,
the expression for $E_r$, Eq.~\eqref{multispec:Er_from_Vphiav},
reduces then to
\bea{multispec:Er_from_Vphiav_limit}
E_r
&=&
\left[ \frac{c \left(B_{\tht_B} + q B_{\ph_B} \right)}{\sqrt{g_B} B^{\ph_B}} \right]^{-1}
\left(
\underbrace{\iota \left(B_{\tht_B} + q B_{\ph_B}\right)}_{= B^2 / B^{\ph_B}}
\left\langle V_\alpha^{\ph} \right\rangle +
\frac{c T_i \left(B_{\tht_B} + q B_{\ph_B} \right)}{e_i \sqrt{g_B} B^{\ph_B}}
\frac{1}{p_i}\difp{p_i}{r} \right. -{} \non
& & -{}
\left.
\frac{c T_i B_{\ph_B}}{e_i \sqrt{g_B} B^{\tht_B}} k(r) \frac{1}{T_i}\difp{T_i}{r}
\right) \non
&=&
\frac{\sqrt{g_B} B^{\tht_B}}{c} \left\langle V_\alpha^{\ph} \right\rangle +
\frac{1}{e_i n_i} \difp{p_i}{r} - \frac{k(r) B_{\ph_B}}
{e_i \underbrace{\left(\iota B_{\tht_B} + B_{\ph_B}\right)}_{\approx \left\langle B^2 \right\rangle / B^{\ph_B}}}
\difp{T_i}{r},
\eea
which agrees with Eq.~(6) [Kasilov et al., Physics of Plasmas 21, 092506 (2014)].\\

If the toroidal rotation frequency is measured at a specific point on a flux surface $(r,\tht_B)$,
activate \verb|isw_Vphi_loc=2| in order to compute the radial electric field.
Fur this purpose the integration constant $K(r)$ can be evaluated in terms of transport coefficients,
which are determined by the NEO-2 solution for the axisymmetric configuration,
\bea{multispec:K_via_VparB}
\left\langle V_{\parallel,\alpha} B \right\rangle
&=&
- \frac{c T_\alpha B_{\ph_B}}{e_\alpha \sqrt{g_B} B^{\tht_B}}
\left( \frac{1}{p_\alpha} \difp{p_\alpha}{r} - \frac{e_\alpha E_r}{T_\alpha} \right)
+ K(r) \left\langle B^2 \right\rangle \non
&\overset{!}{=}&
- \sum\limits_{\alpha'} \left(
D_{31}^{\alpha \alpha'} \left(-\frac{e_{\alpha'} E_r}{T_{\alpha'}}\right) +
D_{31}^{\alpha \alpha'} \frac{1}{p_{\alpha'}}\difp{p_{\alpha'}}{r} +{} \right. \non
& & {}+ \left.
\left( D_{32}^{\alpha \alpha'} - \frac{5}{2} D_{31}^{\alpha \alpha'}\right)
\frac{1}{T_{\alpha'}}\difp{T_{\alpha'}}{r} +
D_{33}^{\alpha \alpha'} \frac{e_{\alpha'}}{T_{\alpha'}}
\frac{\left\langle E_{\parallel} B  \right\rangle }{\left\langle B^2 \right\rangle}
\right),
\eea
as
\bea{multispec:constK}
K(r) &=&
\frac{1}{\left\langle B^2 \right\rangle} \left(
E_r \left(-\frac{c B_{\ph_B}}{\sqrt{g_B} B^{\tht_B}} +
\sum\limits_{\alpha'} D_{31}^{\alpha \alpha'} \frac{e_{\alpha'}}{T_{\alpha'}}\right) +
\frac{c T_\alpha B_{\ph_B}}{e_\alpha \sqrt{g_B} B^{\tht_B}}
\frac{1}{p_\alpha} \difp{p_\alpha}{r} \right. -{} \non
& & {}- \left.
\sum\limits_{\alpha'} \left(
D_{31}^{\alpha \alpha'} \frac{1}{p_{\alpha'}}\difp{p_{\alpha'}}{r} +
\left( D_{32}^{\alpha \alpha'} - \frac{5}{2} D_{31}^{\alpha \alpha'}\right)
\frac{1}{T_{\alpha'}}\difp{T_{\alpha'}}{r} +
D_{33}^{\alpha \alpha'} \frac{e_{\alpha'}}{T_{\alpha'}}
\frac{\left\langle E_{\parallel} B  \right\rangle }{\left\langle B^2 \right\rangle} \right)
\right).
\eea
Using Eq.~\eqref{multispec:def_Gfun}, Eq.~\eqref{multispec:VrCtrvr},
Eq.~\eqref{multispec:VparSpec} and Eq.~\eqref{multispec:constK},
the toroidal rotation frequency is related to the radial electric field via
\bea{multispec:calc_Er_step_loc}
V_\alpha^{\ph} (r,\tht_B)
&=&
V_\alpha^{\ph_B} + 2\pi \psi_{\rm tor}^{\rm a}
\difp{G(s,\tht_B)}{\tht_B} V_\alpha^{\tht_B} \non
&=&
- \frac{c T_\alpha \left(B_{\tht_B} + q B_{\ph_B} \right)} {e_\alpha \sqrt{g_B} B^2}
\left( \frac{1}{p_\alpha} \difp{p_\alpha}{r} - \frac{e_\alpha E_r}{T_\alpha} \right) +
B^{\ph_B} \left( 1 + 2 \pi \iota \psi_{\rm tor}^{\rm a} \difp{G}{\tht_B} \right) K(r).
\eea
This expression can be rearranged to determine the radial electric field as
\bea{multispec:Er_from_VphiLoc}
E_r
&=&
\left[
\frac{c \left(B_{\tht_B} + q B_{\ph_B}\right)}{\sqrt{g_B} B^2} +
\frac{B^{\ph_B}}{\left\langle B^2 \right\rangle}
\left(1 + 2 \pi \iota \psi_{\rm tor}^{\rm a} \difp{G}{\tht_B}\right)
\left(
-\frac{c B_{\ph_B}}{\sqrt{g_B} B^{\tht_B}} +
\sum\limits_{\alpha'} D_{31}^{\alpha \alpha'} \frac{e_{\alpha'}}{T_{\alpha'}}
\right)
\right]^{-1} \times{} \non
& & {}\times
\left(
V_\alpha^{\ph}(r,\tht_B) +
\frac{c T_\alpha}{e_\alpha} \left(
\frac{B_{\tht_B} + q B_{\ph_B}}{\sqrt{g_B} B^2} -
\left(1 + 2 \pi \iota \psi_{\rm tor}^{\rm a} \difp{G}{\tht_B}\right)
\frac{B^{\ph_B} B_{\ph_B}}{\left\langle B^2 \right\rangle \sqrt{g_B} B^{\tht_B}}
\right)
\frac{1}{p_\alpha} \difp{p_\alpha}{r} \right. +{} \non
& & +{}
\frac{B^{\ph_B}}{\left\langle B^2 \right\rangle}
\left(1 + 2 \pi \iota \psi_{\rm tor}^{\rm a} \difp{G}{\tht_B}\right)
\sum\limits_{\alpha'} \left[
D_{31}^{\alpha \alpha'} \frac{1}{p_{\alpha'}}\difp{p_{\alpha'}}{r} +
\left( D_{32}^{\alpha \alpha'} - \frac{5}{2} D_{31}^{\alpha \alpha'}\right)
\frac{1}{T_{\alpha'}}\difp{T_{\alpha'}}{r} \right. +{} \non
& & +{} \left. \left.
D_{33}^{\alpha \alpha'} \frac{e_{\alpha'}}{T_{\alpha'}}
\frac{\left\langle E_{\parallel} B  \right\rangle }{\left\langle B^2 \right\rangle}
\right]
\right),
\eea
where the parallel electric field is assumed to be either known (see next section) or negligible small.

%%%%%%%%%%%%%%%%%%%%%%%%%%%%%%%%%%%%%%%%%%%%%%%%%%%%%%%%%%%%%%%%%%%%%%%%
\subsection{Computation of the parallel electric field}
In the NEO-2 subroutine \verb|compute_A3norm| of \verb|ntv_mod.f90|
the parallel electric field is evaluated from equilibrium currents
specified in the Boozer file and from transport coefficients computed by NEO-2.
The total, parallel equilibrium current is given by following expression
(see next subsection for further details),
\be{multispec:JparB_tot_result}
\left\langle j_\parallel^{\rm tot} B \right\rangle =
\frac{c}{4 \pi} \frac{\sigma_{\sqrt{g}} \left\langle B^2 \right\rangle}
{|\psi_{\rm tor}^{\rm a} \left(B_{\ph_B} + \iota B_{\tht_B}\right)|}
\left(B_{\ph_B} \difp{B_{\tht_B}}{s} -  B_{\tht_B} \difp{B_{\ph_B}}{s}\right),
\ee
where $\sigma_{\sqrt{g}}=\sign(\sqrt{g_B})$ takes into account a
left-handed coordinate system.
This total, parallel equilibrium current can be also expressed in terms of
transport coefficients computed by NEO-2,
\bea{multispec:JparB_tot_Dijab}
\left\langle j_\parallel^{\rm tot} B \right\rangle
&=&
-  \sum\limits_{\alpha,\alpha'} \sum\limits_{k=1}^{3}
e_\alpha n_\alpha D_{3k}^{\alpha\alpha'} A_k^{\alpha'}.
\eea
Using Eq.~\eqref{multispec:JparB_tot_result} and Eq.~\eqref{multispec:JparB_tot_Dijab},
the parallel electric field can be expressed as,
\bea{multispec:avEparB}
\frac{\left\langle E_\parallel B \right\rangle}{\left\langle B^2 \right\rangle}
&=&
\left[
-\sum\limits_{\alpha,\alpha'} e_\alpha n_\alpha
D_{33}^{\alpha\alpha'} \frac{e_{\alpha'}}{T_{\alpha'}}
\right]^{-1}
\left(
\frac{c}{4 \pi} \frac{\sigma_{\sqrt{g}} \left\langle B^2 \right\rangle}
{|\psi_{\rm tor}^{\rm a} \left(B_{\ph_B} + \iota B_{\tht_B}\right)|}
\left(B_{\ph_B} \difp{B_{\tht_B}}{s} -  B_{\tht_B} \difp{B_{\ph_B}}{s}\right)
\right. +{} \non
& & {}+
\left.
\sum\limits_{\alpha,\alpha'} e_\alpha n_\alpha
\left[
-D_{31}^{\alpha\alpha'} \frac{e_{\alpha'}}{T_{\alpha'}} E_r +
D_{31}^{\alpha\alpha'} \difp{\log p_{\alpha'}}{r} +
\left(D_{32}^{\alpha\alpha'}-\frac{5}{2}D_{31}^{\alpha\alpha'}\right)
\difp{\log T_{\alpha'}}{r}
\right]
\right).
\eea
Eq.~\eqref{multispec:avEparB} and Eq.~\eqref{multispec:Er_from_Vphiav} /
Eq.~\eqref{multispec:Er_from_VphiLoc} yield a system of equations
for the parallel and radial electric field which can be solved
iteratively (see subroutine \verb|get_Er| of \verb|ntv_mod.f90|)
or directly (see subroutine \verb|compute_Er_and_A3norm| of \verb|ntv_mod.f90|).

%%%%%%%%%%%%%%%%%%%%%%%%%%%%%%%%%%%%%%%%%%%%%%%%%%%%%%%%%%%%%%%%%%%%%%%%
\subsection{Computation of the parallel equilibrium current from a Boozer file}
The total, parallel equilibrium current can be computed from the
magnetic field data provided by the Boozer file via Ampere's law,
\be{multispec:AmpereLaw}
\nabla \times \bB = \frac{4\pi}{c} \bj.
\ee
Firstly, the averaged toroidal and poloidal contra-variant components of the
current density are calculated from Ampere's law,
\bea{multispec:av_JphiCtrvr and_JthtCtrvr}
\left\langle j^{\ph_B} \right\rangle
&=&
\hphantom{-} \pi c
\left(
\int\limits_{0}^{2\pi} \rd\ph_B \int\limits_{0}^{2\pi} \rd\tht_B \sqrt{g_B}
\right)^{-1}
\difp{B_{\tht_B}}{s}, \non
%
\left\langle j^{\tht_B} \right\rangle
&=&
-\pi c
\left(
\int\limits_{0}^{2\pi} \rd\ph_B \int\limits_{0}^{2\pi} \rd\tht_B \sqrt{g_B}
\right)^{-1}
\difp{B_{\ph_B}}{s}.
\eea
Please note that in this subsection $s$ is used as a flux surface label
(i.e., ($\sqrt{g_B})^{-1}=({\nabla}s \times \nabla\tht_B)\cdot \nabla\ph_B$) which is common choice for Boozer files.
Secondly, the averaged toroidal and poloidal contra-variant components of the
current density are splitted into a parallel and perpendicular component
with respect to the magnetic field line,
\bea{multispec:JphiCtrvr and_JthtCtrvr_parperp}
\left\langle j^{\ph_B} \right\rangle
&=&
\left\langle j_{\parallel}^{\rm tot} h^{\ph_B} \right\rangle +
\left\langle j_{\perp}^{\ph_B} \right\rangle =
\frac{B^{\ph_B}}{B^2}
\left\langle j_{\parallel}^{\rm tot} B \right\rangle +
\left\langle j_{\perp}^{\ph_B} \right\rangle, \non
%
\left\langle j^{\tht_B} \right\rangle
&=&
\left\langle j_{\parallel}^{\rm tot} h^{\tht_B} \right\rangle +
\left\langle j_{\perp}^{\tht_B} \right\rangle =
\frac{B^{\tht_B}}{B^2}
\left\langle j_{\parallel}^{\rm tot} B \right\rangle +
\left\langle j_{\perp}^{\tht_B} \right\rangle.
\eea
Using the ideal MHD force balance equation,
\be{multispec:idealMHDforcebal}
\nabla \times \bB = \frac{4\pi}{c} \bj,
\ee
the perpendicular component of the current density $\bj_\perp$
can be determined,
\be{multispec:jperp_via_MHDforcebal}
\bj_\perp = \frac{c}{B^2} \bB \times {\nabla}p =
\frac{c}{B^2} \difp{p}{s} \bB \times {\nabla}s.
\ee
The perpendicular components of $j^{\ph_B}$ and $j^{\tht_B}$
are then
\bea{multispec:jperp_ph_and_tht}
j_{\perp}^{\ph_B}(s)
&=&
\frac{c}{B^2} \difp{p}{s} \left(\bB \times {\nabla}s\right) \cdot \nabla\ph_B =
-c \difp{p}{s} \frac{B_{\tht_B}}{\sqrt{g_B} B^2}, \non
%
j_{\perp}^{\tht_B}(s)
&=&
\frac{c}{B^2} \difp{p}{s} \left(\bB \times {\nabla}s\right) \cdot \nabla\tht_B =
\hphantom{-}c \difp{p}{s} \frac{B_{\ph_B}}{\sqrt{g_B} B^2}.
\eea
Using Eq.~\eqref{multispec:av_JphiCtrvr and_JthtCtrvr}, Eq.~\eqref{multispec:JphiCtrvr and_JthtCtrvr_parperp}
and Eq.~\eqref{multispec:jperp_ph_and_tht}, $\left\langle j_{\parallel}^{\rm tot} B \right\rangle$
can be computed,
\bea{multispec:JparB_tot_step1}
\left\langle j_{\parallel}^{\rm tot} B \right\rangle
&=&
\pi c
\underbrace{
\left(
\int\limits_{0}^{2\pi} \rd\ph_B \int\limits_{0}^{2\pi} \rd\tht_B \sqrt{g_B}
\right)^{-1}
}_{= \left\langle B^2 \right\rangle /
\left( 4\pi^2 \sigma_{\sqrt{g}} |\psi_{\rm tor}^{\rm a}
\left(B_{\ph_B} + \iota B_{\tht_B}\right)|\right)}
\left(B_{\ph_B} \difp{B_{\tht_B}}{s} -  B_{\tht_B} \difp{B_{\ph_B}}{s}\right) \non
&=&
\frac{c}{4 \pi} \frac{\sigma_{\sqrt{g}} \left\langle B^2 \right\rangle}
{|\psi_{\rm tor}^{\rm a} \left(B_{\ph_B} + \iota B_{\tht_B}\right)|}
\left(B_{\ph_B} \difp{B_{\tht_B}}{s} -  B_{\tht_B} \difp{B_{\ph_B}}{s}\right),
\eea
where $\sigma_{\sqrt{g}}=\sign(\sqrt{g_B})$ takes into account a
left-handed coordinate system.

%%%%%%%%%%%%%%%%%%%%%%%%%%%%%%%%%%%%%%%%%%%%%%%%%%%%%%%%%%%%%%%%%%%%%%%%



\subsection{Output files}
The results for a given flux surface are stored in HDF5 file "neo2\_multispecies\_out.h5".
Contents of this file is described in the tables below.

\begin{table}[h]
\begin{tabular}{|c|c|c|l|}
\hline
\hline
\# & Variable name in HDF file & Dimension  & Comment \\
\hline
\hline
55 & R0 & scalar & Major radius of the magnetic axis [cm],
R0 = rt0 where rt0 = $R_0$.
\\
\hline
80 & boozer\_s & scalar & Normalized toroidal flux (label) [dimensionless],
boozer\_s = $s = \psi_{\rm tor}/\psi^a_{\rm tor}$.
\\
\hline
71 & aiota & scalar & Rotational transform [dmls], aiota = aiota\_loc where
aiota\_loc = $\iota(s)=1/q(s)$.
\\
\hline
%
1 & Bref & scalar & Reference magnetic field [G], Bref = bmod0*1e4 = $B_{\rm ref}$, \\
 & & & Here bmod0 = $B_{00}(s_{\rm ref})$ is assumed to be in SI units, where $B_{00}$ is Boozer (0,0)\\
 & & & harmonic of $B$ at the surface with normalized toroidal flux value $s_{\rm ref}$\\
 & & & where $s_{\rm ref}=0$ for ref\_swi=1 and $s_{\rm ref}=s$ for ref\_swi=3. \\
 & & & Variable ref\_swi is specified in the input file ``neo.in'' .
\\
\hline
77 & avnabpsi & scalar &
Flux surface averaged gradient of the label [1/cm],
avnabpsi = $\langle |\nabla s| \rangle$.
\\
\hline
75 & avbhat & scalar &
Flux surface average normalized magnetic field [dmls],
avbhat = $\langle B / B_{\rm ref}\rangle$.
\\
\hline
74 & av\_inv\_bhat & scalar & Flux surface average inverse normalized magnetic field [dmls],
\\
 & & &
av\_inv\_bhat = av\_inv\_bhat\_val where av\_inv\_bhat\_val = $\langle B_{\rm ref}/B\rangle$.
\\
\hline
76 & avbhat2 & scalar &
Flux surface average normalized magnetic field square [dmls],
\\
 & & &
avbhat2  = $\langle \left(B / B_{\rm ref}\right)^2\rangle$.
\\
\hline
73 & av\_gphph & scalar &
Flux surface average covariant toroidal metric tensor component [cm$^2$],
\\
 & & &
av\_gphph = av\_gphph\_val where av\_gphph\_val = $\langle g_{\varphi\varphi}\rangle$.
\\
\hline
79 & bcovar\_tht & scalar & Covariant poloidal magnetic field component [G cm],
bcovar\_tht = $B_\vartheta(s)$.
\\
\hline
78 & bcovar\_phi & scalar & Covariant toroidal magnetic field component [G cm],
bcovar\_phi = $B_\varphi(s)$.
\\
\hline
84 & dbcovar\_theta\_ds & scalar &
Derivative of covariant poloidal magnetic field component over label [G cm],
\\
 & & &
dbcovar\_theta\_ds = $\partial B_\vartheta(s) / \partial s$.
\\
\hline
83 & dbcovar\_phi\_ds & scalar &
Derivative of covariant toroidal magnetic field component over label [G cm],
\\
 & & &
dbcovar\_phi\_ds = $\partial B_\varphi(s) / \partial s$.
\\
\hline
96 & sqrtg\_bctrvr\_tht & scalar &
Poloidal contra-variant magentic field component times $\sqrt{g}$ [G cm$^2$],
\\
 & & &
sqrtg\_bctrvr\_tht = $\sqrt{g} B^\vartheta$ where $\sqrt{g}$ is metric determinant of
$(s,\vartheta,\varphi)$ variables.
\\
\hline
95 & sqrtg\_bctrvr\_phi & scalar &
Toroidal contra-variant magentic field component times $\sqrt{g}$ [G cm$^2$],
\\
 & & &
sqrtg\_bctrvr\_phi = $\sqrt{g} B^\varphi$ where $\sqrt{g}$ is metric determinant of
$(s,\vartheta,\varphi)$ variables.
\\
\hline
92 & psi\_pr\_hat & scalar & Toroidal flux at the edge normalized by $2\pi B_{\rm ref}$ [dmls],
psi\_pr\_hat = boozer\_psi\_pr\_hat
\\
 & & &
where boozer\_psi\_pr\_hat =
$(\partial \psi /\partial s) /B_{\rm ref} = \psi_{\rm tor}^a/B_{\rm ref} = \Psi_{\rm tor}^a /(2\pi B_{\rm ref})$.
\\
\hline
\hline
\end{tabular}
\caption{Unperturbed magnetic field and flux surface geometry.}
\end{table}






\begin{table}[h]
\begin{tabular}{|c|c|c|l|}
\hline
\hline
\# & Variable name in HDF file & Dimension  & Comment \\
\hline
\hline
%
40 & Dp0 & scalar & Plateau diffusion coefficient [cm$^2$ / s], Dp0=Dp00 where Dp00 = $D_p^\alpha$ Eq.~\eqref{dplateau}\\
 & & & taken for the first species $\alpha=1$ (usually electrons). \\
 & & & Dp00 = PI*vt0*(rho0**2)/(16.0\_dp*aiota\_loc*rt0) \\
\hline
%
30 & D31ref0 & scalar &  Ion-like Ware pinch coefficient \red{[cm$^2$ G / s]}, D31ref0=D31ref00  \\
 & & & where D31ref00 = $D_{13}^{AX,\alpha}$ Eq.~\eq{wareion} taken for $\alpha=1$ (usually electrons). \\
 & & & D31ref00 = vt0 * rho0 * bcovar\_phi\_hat * (bmod0*1e4) / \& \\
 & & & (2 aiota\_loc * boozer\_psi\_pr\_hat * avnabpsi) \\
\hline
%
35 & D33L0\_Zeff & scalar & Rescaled Lorentz conductivity \red{[cm$^2$ G$^2$ / s]}, D33L0\_Zeff = D33L00\_Zeff, \\
 & & & where D33L00\_Zeff = $D_{33}^L$ Eq.~\eq{D33L}.\\
 & & & D33L00\_Zeff = (-16.0\_dp/(3.0\_dp*PI)) * ((bmod0*1.0e4\_dp)**2) * \& \\
 & & & vt0 * (2.0\_dp/collpar\_spec(0)) \\
\hline
\hline
%
3 & D11\_AX\_Dpl & vectors &
Axisymmetric normalized transport coefficients Eq.~\eq{multispec:platnorm} [dimensionless].
\\
7 & D12\_AX\_Dpl & [num\_spec**2] &
Notation: D12\_AX\_Dpl = $D_{12}^{\alpha \alpha'}/D_p^{\alpha_1}$, etc.
Each coefficient is a matrix w.r.t.
\\
15 & D21\_AX\_Dpl & &
species indices $\alpha, \alpha' = 1,\dots,$num\_spec. Matrix is put in a vector
with order
\\
19 & D22\_AX\_Dpl & &
of elements
\red{
$(\alpha,\alpha^\prime)$ = (1,1), (1,2), \dots (2,1), (2,2), \dots, (num\_spec, num\_spec).
}
\\
\hline
%
11 & D13\_AX\_D31ref & vectors &
Axisymmetric normalized transport coefficients Eqs.~\eq{multispec:warenorm}
and~\eq{multispec:bootstrapnorm} [dmls].
\\
23 & D23\_AX\_D31ref &  [num\_spec**2] &
Notation: D23\_AX\_D31ref = $D_{LL^\prime}^{\alpha \alpha'}/ D_{13}^{{\rm AX},\alpha_1}$, etc.
\\
27 & D31\_AX\_D31ref & &
Odering of elements $(\alpha,\alpha^\prime)$ in vectors
is the same
\\
32 & D32\_AX\_D31ref & &
as for D11\_AX\_Dpl  through D22\_AX\_Dpl (see above).
\\
\hline
37 & D33\_AX\_norm & vector &
Axisymmetric normalized transport coefficient Eq.~\eq{multispec:D33norm} [dmls].
\\
 & & [num\_spec**2] &
Notation: D33\_AX\_norm = $D^{\alpha \alpha'}_{33}/\left(Z_{\rm eff} D_{33}^{L}\right)$.
\\
 & & &
Odering of elements $(\alpha,\alpha^\prime)$ in vectors
is the same
\\
 & & &
as for D11\_AX\_Dpl  through D22\_AX\_Dpl (see above).
\\
\hline
\hline
\end{tabular}
\end{table}





\begin{table}[h]
\begin{tabular}{|c|c|c|l|}
\hline
\hline
\# & Variable name in HDF file & Dimension  & Comment \\
\hline
\hline
 & & vectors &
Axisymmetric dimensional transport coefficients Eq.~\eq{multispec:difcoefs}. Ordering of elements
\\
 & & [num\_spec**2] &
over species indices is the same as for axisymmetric normalized coefficients
\\
2 & D11\_AX & & = D11\_AX\_Dpl * Dp0 = $D^{\alpha \alpha'}_{11}$ [cm$^2$ / s]
\\
6 & D12\_AX & & = D12\_AX\_Dpl * Dp0 = $D^{\alpha \alpha'}_{12}$ [cm$^2$ / s]
\\
10 & D13\_AX & & = D13\_AX\_D13ref * D31ref0 = $D^{\alpha \alpha'}_{13}$ [cm$^2$ G / s]
\\
14 & D21\_AX & & = D21\_AX\_Dpl * Dp0 = $D^{\alpha \alpha'}_{21}$ [cm$^2$ / s]
\\
18 & D22\_AX & & = D22\_AX\_Dpl * Dp0 = $D^{\alpha \alpha'}_{22}$ [cm$^2$ / s]
\\
22 & D23\_AX & & = D23\_AX\_D31ref * D31ref0 = $D^{\alpha \alpha'}_{23}$ [cm$^2$ G / s]
\\
26 & D31\_AX & & = D31\_AX\_D31ref * D31ref0 = $D^{\alpha \alpha'}_{31}$ [cm$^2$ G / s]
\\
31 & D32\_AX & & = D32\_AX\_D31ref * D31ref0 = $D^{\alpha \alpha'}_{32}$ [cm$^2$ G / s]
\\
36 & D33\_AX & & = D33\_AX\_norm * D33L00\_Zeff = $D^{\alpha \alpha'}_{33}$ [cm$^2$ G$^2$ / s]
\\
\hline
\hline
5 & D11\_NA\_Dpl & vectors &
Non-axisymmetric normalized transport coefficients Eq.~\eq{multispec:platnorm} [dmls].
\\
9 & D12\_NA\_Dpl &  [num\_spec**2] &
Notation and species ordering are the same as for respective
\\
17 & D21\_NA\_Dpl & &
axisymmetric coefficients D11\_AX\_Dpl, D12\_AX\_Dpl, D21\_AX\_Dpl
\\
21 & D22\_NA\_Dpl & &
and D22\_AX\_Dpl.
\\
\hline
13 & D13\_NA\_D31ref & vectors &
Non-axisymmetric normalized transport coefficients Eqs.~\eq{multispec:warenorm}
and~\eq{multispec:bootstrapnorm} [dmls].
\\
25 & D23\_NA\_D31ref &  [num\_spec**2] &
Notation and species ordering are the same as for respective
\\
29 & D31\_NA\_D31ref & &
axisymmetric coefficients D13\_AX\_D31ref, D23\_AX\_D31ref,
\\
34 & D32\_NA\_D31ref & &
D31\_AX\_D31ref and D32\_AX\_D31ref.
\\
\hline
39 & D33\_NA\_norm & vector &
Non-axisymmetric normalized transport coefficient Eq.~\eq{multispec:D33norm} [dmls].
\\
 &  & [num\_spec**2] & Notation and species ordering is the same as for D33\_AX\_norm
\\
\hline
\hline
%
4 & D11\_NA & vectors &
Non-axisymmetric dimensional transport coefficients Eq.~\eq{multispec:difcoefs}.
\\
8 & D12\_NA & [num\_spec**2] &
Units, notation and species ordering are the same as for respective
\\
12 & D13\_NA & &
axisymmetric coefficients D11\_AX, D12\_AX, \dots, D33\_AX
\\
16 & D21\_NA & &
\\
20 & D22\_NA & &
\\
24 & D23\_NA & &
\\
28 & D31\_NA & &
\\
33 & D32\_NA & &
\\
38 & D33\_NA & &
\\
\hline
\hline
\end{tabular}
%\caption{Contents of the output file ``neo2\_multispecies\_out.h5'', part 1}
\end{table}

\begin{table}[h]
\begin{tabular}{|c|l|}
\hline
variable name in HDF file  & Comment \\
\hline
T\_spec & temperature of the species [erg] \\
\hline
col\_ind\_spec & \\
\hline
collpar\_spec & \\
\hline
\end{tabular}
\caption{Contents of the output file ``neo2\_multispecies\_out.h5'', part 3}
\end{table}

\begin{table}[h]
\begin{tabular}{|c|l|}
\hline
variable name in HDF file  & Comment \\
\hline
eps\_M\_2 & \\
\hline
isw\_coul\_log & \\
\hline
m\_phi & \\
\hline
m\_spec & mass of the species [g] \\
\hline
n\_spec & density of the species [$1/cm^3$] \\
\hline
nu\_star\_spec & \\
\hline
num\_spec & number of species [1] \\
\hline
row\_ind\_spec & \\
\hline
species\_tag & \\
\hline
version & \\
\hline
z\_spec & charges of the species [1]
\end{tabular}
\caption{Contents of the output file ``neo2\_multispecies\_out.h5'', part 4}
\end{table}


\begin{table}[h]
\begin{tabular}{|c|c|c|l|}
\hline
\hline
\# & Variable name in HDF file & Dimension  & Comment \\
\hline
\hline
%
41 & Er & scalar &
Radial electric field [statvolt / cm], Er = $E_r$ Eq.~\eq{multispec:Er_from_VphiLoc}.
\\
\hline
43 & Gamma\_AX\_spec & vector [num\_spec]&
\\
\hline
45 & Gamma\_NA\_spec & &
\\
\hline
 & MtOvR & &
\\
\hline
 & ParFlow\_AX\_spec & &
\\
\hline
 & ParFlow\_NA\_spec & &
\\
\hline
 & Qflux\_AX\_spec & &
\\
\hline
 & Qflux\_NA\_spec & &
\\
\hline
 & R\_Vphi\_prof & &
\\
\hline
 & TphiNA\_spec & &
\\
\hline
 & TphiNA\_tot & &
\\
\hline
 & VphiB\_spec & &
\\
\hline
 & Vphi\_prof\_spec & &
\\
\hline
 & VthtB\_spec & &
\\
\hline
 & Vtht\_prof\_spec & &
\\
\hline
 & Z\_Vphi\_prof & &
\\
\hline
\hline
\end{tabular}
\caption{Contents of the output file ``neo2\_multispecies\_out.h5'',
part 5. These quantities are onyl written if $isw\_calc\_Er=1$.}
\end{table}







\begin{table}[h]
\begin{tabular}{|c|l|}
\hline
variable name in HDF file  & Comment \\
\hline
Gamma\_AX\_Ware\_spec & \\
\hline
Gamma\_NA\_Ware\_spec & \\
\hline
ParFlow\_AX\_Ware\_spec & \\
\hline
ParFlow\_NA\_Ware\_spec & \\
\hline
Qflux\_AX\_Ware\_spec & \\
\hline
Qflux\_NA\_Ware\_spec & \\
\hline
TphiNA\_Ware\_spec & \\
\hline
TphiNA\_Ware\_tot & \\
\hline
VphiB\_Ware\_spec & \\
\hline
Vphi\_prof\_woWare\_spec & \\
\hline
VthtB\_Ware\_spec & \\
\hline
Vtht\_prof\_woWare\_spec & \\
\hline
avEparB\_ov\_avb2 &
\end{tabular}
\caption{Contents of the output file ``neo2\_multispecies\_out.h5'',
part 6. These quantities are only written if $isw\_calc\_Er=1$ and
$num\_spec > 1$.}
\end{table}

%%%%%%%%%%%%%%%%%%%%%%%%%%%%%%%%%%%%%%%%%%%%%%%%%%%%%%%%%%%%%%%%%%%%

\subsection{Perpendicular fluid velocities}

Perpendicular fluid velocity is given in the vector form by Eq.~\eq{multispec:VperpSpec}.
According to~\eq{multispec:VperpSpec_ctrvr}, contra-variant components of this velocity in
Boozer coordinates are flux functions. Let us indroduce a coordinate-free scalar quantity
$V_{\perp,\alpha}$ which would be the same with perpendicular fluid velocity in the straight
cylinder limit as follows
\be{scalvperp}
V_{\perp,\alpha}=\bV_{\perp,\alpha}\cdot\frac{\nabla r \times\bh}{|\nabla r|}
=
-\frac{c}{B} \left( \frac{1}{e_\alpha n_\alpha} \difp{p_\alpha}{r} - E_r \right)
\frac{|\bh \times {\nabla}r|^2}{|\nabla r|}
=\frac{c}{B} \left( E_r - \frac{1}{e_\alpha n_\alpha} \difp{p_\alpha}{r}\right) |\nabla r|.
\ee
This quantity still depends on the poloidal angle and should be averaged as follows
\be{avvperp}
\bar V_{\perp,\alpha}\equiv
\frac{\langle B V_{\perp,\alpha}\rangle}{\langle B \rangle}
=
\frac{c}{\langle B \rangle}
\left( E_r - \frac{1}{e_\alpha n_\alpha} \difp{p_\alpha}{r}\right),
\ee
where we used the definition of radius $\langle |\nabla r| \rangle = 1$.
In addition, we are interested in a modified fluid velocity which is computed
in absence of temperature gradient
\be{avvperp_mod}
\bar V_{\perp,\alpha}^{(mod)}
=
\frac{c}{\langle B \rangle}
\left( E_r - \frac{T_\alpha}{e_\alpha n_\alpha} \difp{n_\alpha}{r}\right),
\ee
Both averaged velocities can be expressed in terms of thermodynamic
forces~\eq{multispec:thermodyn_forces} as follows
\be{avvperp_forces}
\bar V_{\perp,\alpha} = -\frac{c T_\alpha}{e_\alpha \langle B \rangle}
\left(A_1^\alpha+\frac{5}{2}A_2^\alpha\right),
\qquad
\bar V_{\perp,\alpha}^{(mod)} = -\frac{c T_\alpha}{e_\alpha \langle B \rangle}
\left(A_1^\alpha+\frac{3}{2}A_2^\alpha\right).
\ee
Quantities required for evaluation of Eqs~\eq{avvperp} and~\eq{avvperp_mod} are available
in the input file ``neo2.in'' and output file ``neo2\_multispecies\_out.h5''. One should
note that radial derivatives of density and temperature given in the input file as the
derivatives over normalized toroidal flux $s$ are converted to derivatives over
the effective radius $r$ as follows,
\be{radders}
\difp{n_\alpha}{r}=\langle|\nabla s|\rangle \difp{n_\alpha}{s},
\qquad
\difp{T_\alpha}{r}=\langle|\nabla s|\rangle \difp{T_\alpha}{s},
\ee
where $\langle|\nabla s|\rangle$ is given in the output file variable ``avnabpsi''.

\end{document}
