\documentclass{article}
\usepackage{graphicx}
\usepackage{wrapfig}
\usepackage{amsmath}

\usepackage[english]{babel}
\usepackage[utf8]{inputenc}
\usepackage[T1]{fontenc}

\usepackage[a4paper, top=3cm, bottom=3cm]{geometry}
\usepackage{authblk}
\usepackage{setspace}
\usepackage{csquotes}
\usepackage{booktabs}
%%%%%%%%%%%%%%%%%% My preamble %%%%%%%%%%%%%%%%%%%%%%%%%%
%\usepackage[numbers, sort&compress]{natbib}
%\setcitestyle{numbers,square,comma,aysep={,},yysep={,},notesep={,}}
%\bibliographystyle{unsrtnat}
%\bibliographystyle{IEEEtranSN}

%References
\usepackage[
    bibstyle=phys,
    biblabel=brackets,
    citestyle=numeric-comp,
    isbn=false,
    doi=false,
    sorting=none,
    url=false,
    defernumbers=true,
    bibencoding=utf8,
    backend=biber,
    %maxbibnames=3,maxcitenames=3,
    %minnames=3,
    maxnames=30,
    ]{biblatex}
\setcounter{biburllcpenalty}{7000}
\setcounter{biburlucpenalty}{8000}
\addbibresource[]{eccd_ref.bib}
\DeclareBibliographyCategory{fullcited}
\newcommand{\mybibexclude}[1]{\addtocategory{fullcited}{#1}}
\DeclareFieldFormat{titlecase}{\MakeCapital{#1}}
\DeclareFieldFormat{sentencecase}{\MakeSentenceCase{#1}}

\usepackage[font=small,labelfont=bf]{caption}
\usepackage{amssymb}
\usepackage{bm}
\usepackage{listings}

\usepackage{hyperref}

\newcommand{\be}[1]{\begin{equation} \label{#1}}
\newcommand{\ee}{\end{equation}}
\newcommand{\bea}[1]{\begin{eqnarray} \label{#1}}
\newcommand{\eea}{\end{eqnarray}}
\setlength\parindent{0pt}

\newcommand{\br}{{\bf r}}
\newcommand{\bh}{{\bf h}}
\newcommand{\bb}{{\bf b}}
\newcommand{\bv}{{\bf v}}
\newcommand{\bB}{{\bf B}}
\newcommand{\rd}{{\rm d}}
\newcommand{\eq}[1]{(\ref{#1})}
\newcommand{\bp}{{\bf p}}
\newcommand{\difp}[2]{\frac{\partial #1}{\partial #2}}
\newcommand{\iotabar}{\mbox{$\iota$\hspace{-0.365em}-}}
%%%%%%%%%%%%%%%%%%%%%%%%%%%%%%%%%%%%%%%%%%%%%%%%%%%%%

\lstset{basicstyle=\ttfamily}
\newcommand{\vb}{\lstinline}
\newcommand{\vv}[1]{\texttt{\detokenize{#1}}}

\title{\textbf{Using NEO-2-QL for DEMO NTV caclulation\\September 2023}}

\author[1]{Rico~Buchholz}
\affil[1]{Fusion@\"OAW, Institute of Theoretical and Computational Physics, Graz University of Technology, Petersgasse 16, 8010 Graz, Austria}
\renewcommand\Affilfont{\itshape\small}

\date{}

\begin{document}
\onehalfspacing

\maketitle

\section{Introduction}
This serves as documentation on how to do neoclassical toroidal
viscosity calculations or evaluate transport coefficients with
\vv{NEO-2-QL}.
As such it describes the required tasks, but not the
physical background.

Note that the variable \vv{NEO2DATAPATH} used throughout this document,
refers to \vv{/proj/plasma/Neo2/} at the ITCP of TU Graz.

In case of questions regarding this document or \vv{NEO-2-QL}, the authors
can be reached by e-mail\footnote{\makeatletter \url{buchholz@tugraz.at}}.

% ----------------------------------------------------------------------
% ----------------------------------------------------------------------
\section{Short summary}
Below, the required input files and the sequence of all necessary
actions for computation NTV is
summarized. In particular, this is a reminder for  those already
familiar with \vv{NEO-2-QL}. Other usage of \vv{NEO-2-QL} is not described in this
summary but is described in more details in the main text.
\subsection{Required input}
\begin{itemize}
  \item boozer background file
  \item boozer perturbation file
  \item profile file (e.g. density, temperature)
  \item neo.in
  \item neo2.in
  \item (condor submit file|slurm submit file)
\end{itemize}

\subsection{Required steps}
\begin{itemize}
  \item get profiles (Sec.~\ref{running_preparation_profiles})
  \item get magnetic field as boozer file (Sec.~\ref{running_preparation_equilibrium})
  \item make sure you have all required libraries (Sec.~\ref{technicalbackground_libraries})
  \item clone/update \vv{libneo} and \vv{NEO-2} repositories (Sec.~\ref{technicalbackground_git})
  \item compile the codes (Sec.~\ref{technicalbackground_cmake})
  \item create working directory (Sec.~\ref{running_preparation_directories})
  \item get/create configuration files (Sec.~\ref{running_preparation_neo2in})
  \item check/edit settings as necessary (Sec.~\ref{running_preparation_neo2in})
  \item run neo2.x to create subfolders (Sec.~\ref{running_running})
  \item run neo2.x for each subfolder (Sec.~\ref{running_running})
  \item collect data (Sec.~\ref{running_running})
  \item checking output (e.g. plotting) (Sec.~\ref{plotting})
\end{itemize}

% ----------------------------------------------------------------------
% ----------------------------------------------------------------------
\section{Technical background on NEO-2}
\subsection{Overview}
In the last years the development of \vv{NEO-2} has split into two branches,
namely the more general branch for stellarators (and axisymmetric
tokamaks) and the quasilinear version for tokamaks with 3D magnetic
perturbations. The main differences between these two versions at time
of writing this document is given in the following Table~\ref{tab:neo2branches}.

\begin{table}[h]
\centering
\begin{tabular}{lll}
Internal name   & NEO-2-PAR & NEO-2-QL\\
Equilibrium     & Tokamak/Stellarator & Non-axisymmetric tokamak\\
Multiple ion species & No & Yes\\
Relativistic    & Yes & Yes (but not for multispecies)\\
Parallelization & Yes (field line) & Yes (Species)\\
Solver          & $1^\mathrm{st}$ Order & $1^\mathrm{st}$ and $2^\mathrm{st}$ Order%\\
%Distribution function & Yes & Partial\footnote{For plotting only, no interface for fast interpolation}
\end{tabular}
\caption{Comparison of \vv{NEO-2} branches}
\label{tab:neo2branches}
\end{table}

% ----------------------------------------------------------------------
\subsection{Libraries\label{technicalbackground_libraries}}
\vv{NEO-2} depends on a number of external libraries. These libraries
are:
\begin{itemize}
\item SuiteSparse
\item Metis
\item SuperLU
\item GSL
\item FGSL
\item HDF5 - interfaced via hdf5tools
\item OpenMPI - interfaced via MyMPILib
\item libneo
\end{itemize}

SuiteSparse is used for solving the sparse linear system of differential
equations in the ripple solver. Metis is used by SuiteSparse for speedup
of some specific routines and is not obligatory. SuperLU can be used
instead of SuiteSparse by a switch in the input file.

The GSL (Gnu Scientific Library) and its Fortran Interface FGSL are used
for several purposes. The first one provides very efficient numerical
integration methods used for the computation of the matrix elements of
the collision operator in \verb|collop_compute.f90|. Additionally, GSL
provides B-spline routines, also used for the collision operator module
(basis functions). While \vv{gsl} usally installed on (scientific) linux
systems, \vv{fgsl} is not. You thus probably need to compile it on your
own. Be sure to select an \vv{fgsl} version that is suitable for the
\vv{gsl} version used.

The HDF5 library is required for modern I/O of \vv{NEO-2}. One advantage over
text files is that each variable (dataset) has a unique name and can
store additional attributes such as units and/or comments. The usage of HDF5
was unavoidable for storing the large datasets that occur when computing
the generalized Spitzer function for stellarators ($\gtrsim 10\mathrm{GB}$ with \vv{hdf5}
for a single flux surface). In order to simplify
the calls a collection of wrapper routines has been created. This
project has grown over the time so that it was decided to use these
wrapper functions also in other projects, as in the interface, so that
it became an own library called \vv{hdf5tools}.

For parallelization the Message Passing Interface (MPI) is used. The
usage of MPI in \vv{NEO-2} is performed via an own library, called MyMPILib,
and is not restricted to a particular MPI implementation. At our
institute it is linked against OpenMPI, where on clusters Intel MPI is
mostly used. MyMPILib was developed so that in the code of \vv{NEO-2} no
native MPI commands have to be used.

Note that it was decided to no longer develop \vv{hdf5tools} and
\vv{MyMPILib} separate, but to add them to \vv{libneo},
which is intended to collect code that is used by multiple programs of
the group. To be able to use only what is needed, compilation is still
done into seperate library files. You can get \vv{libneo} from github (see
Sec.~\ref{technicalbackground_git}).

% ----------------------------------------------------------------------
\subsection{The Git repository\label{technicalbackground_git}}
Since 2013 the general version of \vv{NEO-2} is under Git version control.
Later, also the quasilinear version was added to this repository
including its code history. The official repository is located on github
at \url{https://github.com/itpplasma/NEO-2/}.
At the moment it is set to private, which means you need to be logged in
to be able to see it.

In order to start a new local working copy, it is required to do
\begin{verbatim}
git clone git@github.com:itpplasma/NEO-2.git .
\end{verbatim}
in an empty directory (do not oversee the point at the end of the
command line which defines the current directory).

The same holds for \vv{libneo}, \url{https://github.com/itpplasma/libneo/},
which is required by \vv{NEO-2}. Here the command is
\begin{verbatim}
git clone git@github.com:itpplasma/libneo.git .
\end{verbatim}
which should be done in another empty directory.

Once you have cloned \vv{libneo} and \vv{NEO-2} you should set the
variables \vv{LIBNEOPATH} and \vv{NEO2PATH} in your \vv{.bashrc} file.
In case you used above commands in your home folder the required lines
could read
\begin{verbatim}
export NEO2PATH=${HOME}/NEO-2/
export LIBNEOPATH=${HOME}/libneo/
\end{verbatim}
Variable \vv{NEO2PATH} is also used throughout this document.

Three other variables that you might want to adapted are \vv{PATH},
\vv{PYTHONPATH} and \vv{MATLABPATH}. Adding the script paths of \vv{NEO-2}
and \vv{libneo} to \vv{PATH} allows you to use the scripts within without
having to type the full path or to copy/link them.
The other two variables make sure python/matlab, respectively will find
the corresponding scripts.
For octave you can add a line
\begin{verbatim}
addpath('/path/to/NEO-2/OctaveScripts/', '-begin')
\end{verbatim}
in file \vv{.octaverc} (not sure if environment variables work in this
file, i.e. if \vv{NEO2PATH} can be used).

% ----------------------------------------------------------------------
\subsection{CMake and Compiling \label{technicalbackground_cmake}}
The build system of \vv{NEO-2} and \vv{libneo} is CMake.
The main advantage is that this build system takes care of correct
compilation order of the source files (dependency resolving) and that
routines can be defined to automatically detect the installed
libraries. For a range of often used libraries there are already routines
available. Using
CMake the source- and build-directories can be separated from each
other, and both codes require that this is done.

The configuration file for \vv{cmake} is called \verb|CMakeLists.txt|.
In order to build
the source code it is necessary to create a build-directory. For reasons
of readability it was decided to put the source files to be compiled in
a separate file, called \vv{CMakeSources.in}. Additionaly, some of the
settings for paths are in a separate file, \vv{ProjectConfig.cmake.in}.
Mainly these settings are related to required libraries (see
sec.~\ref{technicalbackground_libraries}).
These files are then included in \vv{CMakeLists.txt}.

For our purpose a build-directory is now created for \vv{libneo}:
\begin{verbatim}
mkdir Build
cd Build
cmake ..
make
\end{verbatim}

Running the command \verb|cmake ..| is only necessary the first time a
new build directory is used. This command needs to know where
\verb|CMakeLists.txt| is located (therefore the \verb|..|). A lot
happens when this command is called, the most important steps are the
Fortran compiler and library detection. A standard Makefile is generated
which can be used by running \vv{make}.

After libneo is compiled, you can compile \vv{NEO-2-QL}:
\begin{verbatim}
cd NEO-2-QL
mkdir Build
cd Build
cmake ..
make
\end{verbatim}

The first step is required as the two versions of \vv{NEO-2} have
separate folders and separate build-configuration files.

% ----------------------------------------------------------------------
% ----------------------------------------------------------------------
\section{Running an NTV computation}
\subsection{Preparations}
To run \vv{NEO-2} with a Boozer file as input for the magnetic field,
the following input files are required:
\begin{itemize}
 \item \vv{neo.in} (see \ref{running_preparation_neoin})
 \item \vv{neo2.in} (see \ref{running_preparation_neo2in})
 \item Boozer files (*.bc) for background and perturbation (see \ref{running_preparation_equilibrium})
 \item profile file (see \ref{running_preparation_profiles})
\end{itemize}


\subsubsection{neo.in\label{running_preparation_neoin}}
The first line of this file indicates the name of the background boozer
file. Other parameters are e.g. related to interpolation of the magnetic
field module and to the code NEO.

\subsubsection{neo2.in\label{running_preparation_neo2in}}
This file is a Fortran namelist file. It contains physical input
quantities, as well as numerical and technical parameters. Here, the
most important parameters are described.
Check also \vv{neo2.in.ql-full} in the DOC folder of the \vv{NEO-2}
repository. The file should contain an up-to-date list of all the input
parameters together with default values and maybe a short explanation.

Note on naming convention: the prefix lsw and isw stand for logical
switch and integer switch, respectively.

\begin{itemize}
 \item \verb|boozer_s| \newline
 Defines the flux surface as normalized toroidal flux.
 \item \verb|conl_over_mfp|\newline
 This is the collisionality parameter. When provided positive it is
 $L_c/l_c$ (connection length over mean free path) and when provided as
 negative value it is $\kappa = 2/l_c$. Please be aware that in our
 papers we define $\kappa = 1/l_c$, while internally in \vv{NEO-2} it has a
 slightly different normalization.
 \item \verb|lag|\newline
 Number of basis functions
 \item \verb|leg|\newline
 Number of Legendre polynomials
 \item \verb|fname_multispec_in|
 Name of the profile file. Used when creating the subfolders for flux
 surfaces.
 \item \verb|v_max_resolution|\newline
 Only affects level placement and defines the maximum normalized
 velocity that should be resolved by the grid. Experience showed that
 values of $2$ - $3$ are sufficient for reconstruction of the
 generalized Spitzer function up to $5$ times the thermal velocity.
 \item \verb|collop_base_prj|\newline
 Projection base for basis function expansion.
 \begin{itemize}
  \item 0: Generalized Laguerre polynomials of order $3/2$ (Default).
  \item 1: Standard polynomials $\phi_m(x) = x^m$.
  \item 2: Quadratic polynomials $\phi_m(x) = x^{2m}$.
  \item 10: Cubic Splines generated from a $y_m = (0, 0, ..., 1, ..., 0)$ grid.
  \item 11: General B-Splines (best choice).
 \end{itemize}

 \item \verb|collop_base_exp|\newline
 Expansion base for basis function expansion. See \verb|collop_base_prj|
 for parameters. At the moment it is only tested for
 \verb|collop_base_prj = collop_base_exp|.

 \item \verb|collop_bspline_order|\newline
 According to the B-Spline definition this is the order parameter $k$.
 As an example $k=3$ creates quadratic B-Splines and $k=4$ cubic
 B-Splines (best choice).

 \item \verb|phi_x_max|\newline
 Important parameter for numerical integration and definition of
 B-Spline knot distribution. The B-Splines are distributed between
 $x=0$ and this value. A typical choice is $5$. Above this value the
 B-Splines are extrapolated with a Taylor series.

 \item \vv{mag_write_hdf5}\newline
 Creates \vv{magnetic.h5} which contains all information of the magnetic
 field as it is ``seen'' from \vv{NEO-2}.

 \item \vv{in_file_pert}\newline
 Name of the boozer file with the perturbation.
\end{itemize}

% ----------------------------------------------------------------------
\subsection{Preparation of equilibrium\label{running_preparation_equilibrium}}
Equilibrium magnetic field in Boozer coordinates is provided for NEO-2
in the form of text files called in this document ``Boozer files'', with usual file ending ``bc''.
They contain data in a specific format (meant here: not just a table of values).
Examples can be found in \vv{DOC/example_bg_field.bc} and \vv{DOC/example_pert_field.bc}.

Equilibrium files might be generated from \vv{VMEC}. In this case they need
to be converted, as file format and coordinate system does not match
the requirements of \vv{NEO-2}.
There are multiple ways to do this conversion, but we will here mention
only two. The python file \vv{boozer.py} from \vv{libneo} contains a
function to do this and a class with methods to do this. The easier
way is to use the file as a script while providing the name of the \vv{VMEC}
and optionally a multiplier for the internal grid.

\begin{verbatim}
$LIBNEOPATH/python/libneo/boozer.py wout_test.nc 10
\end{verbatim}

% ----------------------------------------------------------------------
\subsection{Preparation of the plasma parameter profiles\label{running_preparation_profiles}}
Profiles are usually provided as text files, e.g.
\begin{verbatim}
$NEO2DATAPATH/Interface/Profiles/w7x-m111-b3-i1/prf.txt
\end{verbatim}
You may also get them directly from a database.
These
plasma parameter profiles are used for computation of the collisionality
parameter $\kappa$, which together with temperature and density profiels
is used as an input to \vv{NEO-2-QL}.
This input must be in the form of an hdf5 file.

% ----------------------------------------------------------------------
\subsection{Preparation of the run directories\label{running_preparation_directories}}
Next step is the preparation of the flux surface grid for the \vv{NEO-2-QL}
runs.
This done with the profiles input file and the \vv{NEO-2-QL} executable.

It is now necessary to copy the following required files to a new
empty directory (the working directory),
\begin{verbatim}
boozer files
profile file
neo.in configuration file
neo2.in configuration file
NEO-2-QL executable
optionally a condor/slurm submit file
\end{verbatim}

Running the \vv{NEO-2-QL} executable creates the folders for the flux
surfaces, fills them with the required files and replaces the profile
values in neo2.in.

Subfolders will have a name \vv{es_*} where ``*'' will be a fixed point
representation (with ``p'' instead of ``.'') of the flux surface label
(e.g. \vv{es_0p22044}).

\subsubsection{Notes on resources}
The number of processors must match the number of species for the run.

The condor submit files determine the resources requested by the job in
the two fields \vv{request_cpus} (note that this number should match
the number for \vv{-np} in \vv{arguments}) and \vv{request_memory}.
The amount of memory required is fixed by the parameters of the jobs
(you may still request more to make sure they run on a specific subset
of nodes, but there are other options for this).

% ----------------------------------------------------------------------
\subsection{Running the jobs\label{running_running}}
The easiest way to start simulations for all the flux surfaces is to use
the condor system. There is an example submit file,
\vv{DOC/condor_submit_example}, available.
The example makes use of a glob pattern, to submit a job for each
subfolder starting with \vv{es_}.
Using condor can be done by executing the command
\begin{verbatim}
condor_submit name_of_condor_submit_file
\end{verbatim}
from the working directory.
The command will report the number of submitted jobs.
The status of the jobs (waiting, running, finished) can be checked with
\vv{condor_q}.

Once the jobs have finished, in the working directory, the python
commands
\begin{verbatim}
import hdf5tools
hdf5tools.collect_and_reshape('./', 'neo2_multispecies_out.h5', 'neo2_multispecies_out.h5')
\end{verbatim}
should be run. This will collect all the \vv{neo2_multispecies_out.h5}
output files in the subfolders into a single output file (here with the
same name) in the working directory.
(Note: a temporary file is created, which so far needs to be removed
manually).
This should make it easier to work with the data.
In normal cases the collection of data should not take more than a few
minutes.

% ----------------------------------------------------------------------
% ----------------------------------------------------------------------
\section{Plotting data\label{plotting}}
The result of the simulations is stored in an hdf5 file.
Octave is capable of loading these files directly.
Matlab requires a specialised load routine.
For python \vv{NEO-2} has the \vv{hdf5tools} module, with wrapper for
the load routines, as well as functions for specific tasks (although at
the moment not many for post-processing).
Once the data is loaded, it can be post-processed and plotted as usual.


% ----------------------------------------------------------------------
% ----------------------------------------------------------------------
\section{Open tasks}
\begin{itemize}
  \item Include NBI source term.
  \item Distribution function plotter in NEO-2-QL.
  \item Cleanup CMake files.
\end{itemize}

% ----------------------------------------------------------------------
% ----------------------------------------------------------------------
\section{More useful scripts}
\begin{itemize}
  \item \vv{hdf5struct.m} Converts HDF5 file to MATLAB structure. Note,
  that this does not work with octave, but is not required, as \vv{load}
  in octave loads a hdf5 file and converts it to a structure.
  \item \vv{plot_levels.m} Plots the field module along the field line
  with level distribution (needs \vv{magnetics.h5}).
\end{itemize}

%\clearpage
%\renewcommand{\bibname}{References}
%\printbibliography[notcategory=fullcited]

\end{document}
